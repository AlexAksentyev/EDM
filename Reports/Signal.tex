\documentclass{article}

\usepackage{repsty}

\newcommand{\uvec}[1]{\boldsymbol{\hat{#1}}}
\newcommand{\abs}[1]{\vert#1\vert}
\newcommand{\proj}[2]{\pi_{\uvec{#2}}\vec{#1}}

\newcommand{\Dw}{\Delta\omega}
\newcommand{\ii}{\imath}
\begin{document}

\section{Projection of polarization}
\begin{align*}
	\vec{P} &= \sum_i \vec{s}, \\
	\proj{s}{y} &\equiv \uvec{y}\cdot\vec{s} = \abs{\vec{s}}\cos\Theta, \\
	\proj{P}{y} &= \sum_i \proj{s_i}{y} = \abs{\vec{s}}\sum_i \cos\Theta_i, \\
	\Theta_i &= \omega_i\cdot t + \phi_i,~ f(\phi_i),~g(\omega_i).
\end{align*}
$f$ and $g$ are distributions.

The phase $\Theta_i$ is a random variable, hence $\proj{P}{y}$ is a sum of random variables; this sum is distributed normally, with expectation $\Xpct{\proj{P}{y}}[t] = \abs{\vec{s}}\cdot n_b\cdot \int_\infty\cos\theta\cdot f(\theta|t)\td\theta$.

\section{Observations}
Modeling shows that the \textbf{frequency} is affected (it's slightly lower when the $\omega$ distribution is right-skewed, and slightly higher if the opposite is true; a bimodal distribution makes the signal's power spectrum bimodal as well).



\end{document}