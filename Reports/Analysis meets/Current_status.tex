\documentclass[pdf]{beamer}


\title{Modeling of spin-orbital dynamics in a storage ring}

\begin{document}
	\begin{frame}
		\titlepage
	\end{frame}		
\section{What's done}
	\begin{frame}
		\frametitle{Python code}
		\begin{itemize}
			\item Integrates the equations of spin-orbital dynamics with scipy::odeint;
			\item Classes defining most commonly utilized accelerator elements (dipoles, quadrupoles, Wien filters, etc);
			\item Two versions of element positioning imperfections (tilting):
			\begin{itemize}
				\item via computing the tilt matrix, and applying it to the computed field at run time (more general but time-consuming, doesn't preserve guiding field strength by default);
				\item customized tilting for dipole, WF (less time-consuming, preserves the Lorentz force acting on the particle), and shift for quadrupole;
			\end{itemize}
			\item Made up of three major classes:
			\begin{itemize}
				\item[Lattice] for element management (e.g., tilting, section-plots);
				\item[Tracker] handles integration proper, data backup into an .hdf5 file;
				\item[Log] output of Tracker::track; based off numpy::recarray, handles plotting.
			\end{itemize}
		\item Vectorized RHS computation.
		\end{itemize}
	\end{frame}

	\begin{frame}
		\frametitle{Example plots}
		\begin{columns}
			\column{.5\textwidth}
			\begin{itemize}
				\item $S_x \sim s$ in FS lattice w/o tilt
				\item $S_x \sim s$ in FS structure w/ all elements tilted about $\hat s$
				\item $S_y \sim s$ in FS structure w/ elements randomly tilted Norm(0, $10^{-4}$ rad)
			\end{itemize}
			\begin{minipage}{.3\textheight}
				\includegraphics[scale=.15]{Sy_300_sigma_tilts_0057_deg_piece}
			\end{minipage}
			\column{.5\textwidth}
			\begin{minipage}{.3\textheight}
				\includegraphics[scale=.15]{Sx_300_no_tilt}
			\end{minipage}
			\begin{minipage}{.3\textheight}
				\includegraphics[scale=.15]{Sx_300_all_tilts_3_6}
			\end{minipage}
		\end{columns}
	\end{frame}

	\begin{frame}
		\frametitle{C++ code}
		\begin{itemize}
			\item Python not fast enough; a Frozen spin lattice of 397 elements takes 2 secs/turn to run. (Integration takes $3-7\cdot 10^{-3}$ secs/element, depending on the field complexity.)
			\item Rewrote the program core in c++ with Boost::Odeint's integrator and Eigen Matrix for field and state type.
			\item Still speed and precision problems:
			\begin{itemize}
				\item vectorizing code (via VexCL, most likely);
				\item step-size control (this'll probably need redesigning state-type).
			\end{itemize}
		\end{itemize}
	\end{frame}

\section{TODO}
	\begin{frame}
		\frametitle{Tasks from supervisor}
		\begin{itemize}
			\item Study effects of WF tilts (preserves Lorentz force) in FS lattice on $S_x, S_y, S_z$;
			\item Same for quadrupole shifts (doesn't preserve LF);
			\item Study decoherence as a function of the inital beam distribution ($x, y, \delta W$);
			\item Study optimal sextupole placement for the suppression of decoherence and chromaticity;
			\item Modeling of field calibration by effective gamma in the horizontal plane (CW/CCW procedure);
		\end{itemize}
	\end{frame}

	\begin{frame}
		\frametitle{Decoherence test histograms}
		\begin{itemize}
			\item 100 turns; 70 trials; 20-particle bunches: $x \sim N(0, 10^{-3})$, $y \sim 10^{-3}$, $dK \sim N(0, 10^{-4})$
		\end{itemize}
		\begin{figure}
			\includegraphics[scale=.25]{Sy_hist_dK}
			\caption{Histogram from $dK \sim N(0, 10^{-4})$ test}
		\end{figure}
	\end{frame}
	\begin{frame}
		\begin{figure}
			\includegraphics[scale=.3]{Sy_hist_x}
			\caption{Histogram from $x \sim N(0, 10^{-3})$ test}
		\end{figure}
	\end{frame}
	\begin{frame}
		\begin{figure}
			\includegraphics[scale=.3]{Sy_hist_y}
			\caption{Histogram from $y \sim N(0, 10^{-3})$ test}
		\end{figure}
	\end{frame}

\end{document}
