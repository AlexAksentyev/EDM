\documentclass{beamer}
\usepackage[backend=biber,style=authortitle-ibid]{biblatex}
\addbibresource{refs.bib}
\usepackage{url}
\usepackage[makeroom]{cancel}

\newcommand{\w}{\omega}
\newcommand{\avg}[1]{\langle{#1}\rangle}
\newcommand{\nbar}{\bar n}

\title{Comparison of Frozen Spin-type EDM search methods}
\author[shortname]{Yury Senichev \inst{1} \and \underline{Alexander Aksentev} \inst{2,3}}
\institute[shortinst]{\inst{1} Institute for Nuclear Research of RAS \and%
  \inst{2} Forschungszentrum J\"ulich \and%
  \inst{3} NRNU ``MEPhI''
}
\date{19 February 2019}

\begin{document}
\begin{frame}
  \titlepage
\end{frame}

\begin{frame}\frametitle{Considered methods}
  \begin{itemize}
  \item BNL Frozen Spin
  \item I.Koop's Spin Wheel
  \item Frequency Domain Method
  \end{itemize}
\end{frame}

\begin{frame}\frametitle{BNL FS}
  \begin{itemize}
  \item Observation of the vertical polarization component~\footcite[p.~9]{BNL:Deuteron2008}
    $\Delta P_V \approx P\cdot\w_{EDM}\cdot t$ (making it a Space Domain method)
  \item Cross section asymmetry $\varepsilon_{LR}\approx 5\cdot 10^{-6}$ for
    smallest practical values of (horizontal plane) $\w_{MDM}$~\footcite[p.~18]{BNL:Deuteron2008}
  \item[*] Challenging task for polarimetry~\footcite[p.~6]{Mane:SpinWheel}
  \end{itemize}
\end{frame}

\begin{frame}\frametitle{BNL FS}\framesubtitle{Systematics}
  \begin{itemize}
  \item Only known first-order systematic effect pertaining to the spin dynamics is the existence of
    $\avg{E_V}\neq 0$~\footcite[p.~10]{BNL:Deuteron2008}
  \item Error frequency $\w_{syst} \approx \frac{\mu\avg{E_V}}{\beta c\gamma^2}$ changes sign when reversing
    the beam circulation direction (CW/CCW)~\footcite[p.~11]{BNL:Deuteron2008}
  \item However, at practical values of element alignment error, $\w_{syst} \gg \w_{EDM}$, hence
    $P_V = P\frac{\w_{EDM}}{\w}\sin(\w t + \Theta_0) \cancel{\approx}P\w_{EDM} t$;
    a Space Domain method is inapplicable under such conditions
  \item[*] At $\avg{E_V}\rightarrow 0$, Space Domain methods are vulnerable to
    the geometric phase error~\footcite[p.~6]{BNL:Proton}
  \end{itemize}
\end{frame}

\begin{frame}\frametitle{Geometric phase error}
  \begin{itemize}
  \item Caused by the non-commutativity of rotations
  \item Formulated in the angular momentum language, it means the \emph{absence of a definite orientation} of
    the spin precession axis (SPA): $\nbar \rightarrow 0$
  \item[*] Call that the \emph{3D Frozen Spin} state
    \item 3D FS is unstable: any stray magnetic field can tilt the precession plane
  \end{itemize}
\end{frame}

\begin{frame}\frametitle{FS-type methodology}\framesubtitle{Conditions of success}
  \begin{itemize}
  \item One must always have a definite direction of the SPA
  \item Measurements must be done in the frequency domain
  \end{itemize}
  These conditions are satisfied by two methods:
  \begin{itemize}
  \item I.Koop's ``Spin Wheel''
  \item Y.Senichev's ``Frequency Domain''
  \end{itemize}
  (Both of which belong to the Frequency Domain category.)
\end{frame}

\begin{frame}\frametitle{Spin Wheel}
  The Spin Wheel is great; it satisfies both success conditions.
  \begin{itemize}
  \item Apply a radial magnetic field of strength $B_x$ sufficient to turn the spin vector about
    the $\hat x$-axis with a frequency of 1 Hz
  \item $\w_{B_X} \parallel \w_{EDM}$ hence $\w_{net} \propto \w_{EDM} + \w_{B_X}$~\footcite[p.~6]{Mane:SpinWheel}
  \item EDM effect $\hat\w_{EDM} = \frac12\left[\w_{net}(+B_X) + \w_{net}(-B_X)\right]$
  \item Value of $B_X$ is calibrated by measuring the vertical orbit splitting
  \end{itemize}
\end{frame}

\begin{frame}\frametitle{Spin Wheel}\framesubtitle{The good, the bad, the ugly}
  \begin{itemize}
  \item Higher polarization growth rate greatly simplifies the task for polarimetry
  \item Magnetic field calibration by means of orbit split measurements seems unfeasible
  \item Element misalignment-induced error is not accounted for:
    \begin{align*}
      \hat\w_{EDM} &= \frac12\left(\w_{EDM} + \cancel{\w_{B_X}} + \w_{mis} + \w_{EDM} - \cancel{\w_{B_X}} + \w_{mis} \right) \\
      &= \w_{EDM} + \w_{mis}
    \end{align*}
  \end{itemize}
\end{frame}

\begin{frame}\frametitle{Frequency Domain Method}
  This methodology has been developed specifically to deal with misalignment error.
  \begin{itemize}
  \item No reason to apply an external B-field; misalignment $B_X$-field provides a sufficiently fast wheel
  \item The FS condition ensures that $\w_{net} \propto \w_{EDM} + \w_{mis}$
  \item The same EDM estimator $\hat\w_{EDM} = \frac{\w_{net}(+B_X) + \w_{net}(-B_X)}{2}$
  \item To flip the sign of $B_X$ one must reverse the guide field polarity (CW/CCW comeback)
  \item The value of $B_X$ is calibrated via horizontal plane precession frequency
  \end{itemize}
\end{frame}

\begin{frame}
  \begin{center}
    Thank you!
  \end{center}
\end{frame}

\begin{frame}\frametitle{Doubly-magic ring}\framesubtitle{Fundamental assumptions}
  \begin{enumerate}
  \item Both beams are at Frozen Spin: $\w = \w_X = \w_{EDM} + \w_{\avg{B_r}}$
  \item EDM of the secondary beam $\ll$ EDM of the primary beam:
    $\w_{EDM}^{PRI} \gg \w_{EDM}^{SEC}\rightarrow 0 \Rightarrow \w_X^{SEC} \approx \w_{\avg{B_r}}^{SEC}$;
    \label{itm:small_EDM}
  \item Beams on the same design orbit $\Leftrightarrow$ experience same fields:
    $\avg{B_r}^{PRI} = \avg{B_r}^{SEC}$\label{itm:same_orbit}
  \item[*] MDM's of both beams are known to high precision (what for?)
  \end{enumerate}
\end{frame}
\begin{frame}\frametitle{D-M Ring}\framesubtitle{Addressing initial objections}
  \begin{itemize}
  \item Precession frequency difference (given~\ref{itm:small_EDM}):
    $\w_X^{PRI} - \w_X^{SEC} \approx \w_{EDM}^{PRI} + \w_{\avg{B_r}}^{PRI} - \w_{\avg{B_r}}^{SEC}$
  \item[Objection] (to assumption~\ref{itm:same_orbit}): The beams have different mass $\Leftrightarrow$
    $\avg{B_r}^{PRI} = \avg{B_r}^{SEC} \cancel{\Rightarrow}\w_{\avg{B_r}}^{PRI} = \w_{\avg{B_r}}^{SEC}$
  \item Using the Koop Wheel, $\w_X^{SEC} = 0 = \w_{\avg{B_r}}^{SEC} \Rightarrow \avg{B_r}^{SEC} = 0$
    (again require~\ref{itm:small_EDM})
  \item Given the design orbit is shared by both beams, $\w_{\avg{B_r}}^{PRI}$ is also 0, b/c
    $\forall m,\gamma,G\left[\w_{\avg{B_r}}=\frac qm\gamma G\avg{B_r} = 0\Leftrightarrow \avg{B_r}=0\right]$
  \item Sameness of the design orbits is guaranteed by the equation:
    $p^4 - 2\mathcal B p^3 + (\mathcal B^2-\mathcal E^2)p^2 - \mathcal E^2m^2 = 0$,\\ where
    $\mathcal B=qcB_0r_0$, $\mathcal E=qE_0r_0$, $(E_0, B_0, r_0)$ are defined by the primary beam FS condition
  \end{itemize}
\end{frame}
\begin{frame}\frametitle{D-M Ring}\framesubtitle{Main objection}
  \begin{itemize}
  \item But by nulling $\w_{\avg{B_r}}^{PRI/SEC}$ we go to the unstable 3D FS state
  \item Which also forces us back to the Space Domain, since $\w_X^{PRI} \approx \w_{EDM}^{PRI} \ll 1$
  \item Thus, both the FS success conditions are violated
  \item[Concl'n] D-MR solves the machine imperfection fields problem, but, other than that,
    inherits all of the original BNL FS weaknesses
  \end{itemize}
\end{frame}
\begin{frame}\frametitle{Universal SR EDM measurement problems}\framesubtitle{And their canonical solutions}
    \begin{minipage}[t]{.5\linewidth}
    \underline{\textbf{Solved by Spin Wheel}}
    \begin{itemize}
    \item Stray fields
    \item Betatron motion
      \item[*] Both cause variation of $\nbar$
    \end{itemize}
  \end{minipage}~%
  \begin{minipage}[t]{.5\linewidth}
    \underline{\textbf{Solved elsewise}}
    \begin{itemize}
    \item Spin decoherence
      %% \item Variation of the injection point (?)
    \item[Sol'n]: Sextupole fields
    \item Machine imperfections 
    \item[Sol'n]: CW/CCW injection
    \end{itemize}
  \end{minipage}
  %% Of course, problems requiring the Spin Wheel solution automatically force us to the Frequency Domain.
\end{frame}

\end{document}
