
\subsection{The T-BMT equation}\label{sec:TBMT_introduction}
The Thomas-Bargmann-Michel-Telegdi equation describes the dynamics of a spin vector $\vec s$ in
a magnetic field $\vec B$ and electrostatic field $\vec E$. Its generalized version,
which includes the EDM effect, can be written as (in the beam rest frame):~\cite[p.~6]{Eremey:Thesis}
\begin{subequations}
  \begin{align}
    \ddt{\vec s} &= \vec s\times \bkt{\vec\W_{MDM} +\vec\W_{EDM}}, \label{eq:TBMT_main}
    \intertext{where the MDM and EDM angular velocities $\vec\W_{MDM}$ and $\vec\W_{EDM}$ }
    \vec\W_{MDM} &= \frac qm \bkt*{G\vec B - \bkt{G - \frac{1}{\gamma^2-1}}\frac{\vec E\times\vec\beta}{c}},\label{eq:TBMT_MDM} \\
    \vec\W_{EDM} &= \frac qm \frac\eta2 \bkt*{\frac{\vec E}c + \vec\beta\times \vec B}.\label{eq:TBMT_EDM}
  \end{align}
\end{subequations}
In the equations above, $m,~q,~G=(g-2)/2$ are respectively the particle mass, charge,
and anomalous magnetic moment; $\beta = \sfrac{v_0}{c}$, is its relative velocity factor;
$\gamma$ its Lorentz factor. The EDM factor $\eta$ is defined by $d = \eta\frac{q}{2mc}s$, where
$d$ is the particle EDM, $s$ its spin.

In the standard formalism it is usual to operate with the (rotational)
one-turn spin transfer matrix:~\cite[p.~4]{COSY:SpinTuneMapping}
\[
\bold{t}_R = \exp\bkt{-i\pi\nu_s\vec\sigma\cdot\bar n} = \cos\pi\nu_s - i (\vec\sigma\cdot\bar n)\sin\pi\nu_s,
\]

where $\nu_s = \sfrac{\W_s}{\W_{cyc}}$, the ratio of the partile's spin precession frequency
to its cyclotron frequency, is termed \emph{spin tune}, $\bar n$ defines the spin precession axis,
and is called the \emph{invariant spin aixs}.

\subsection{Frozen spin concept}
From equation~\eqref{eq:TBMT_MDM} one can see that, in the absence of an EDM,
the direction of a particle's spin vector can be fixed relative its momentum vector, i.e.
$\vec\W_{MDM}=\vec 0$; in other words, one can realize the Frozen Spin condition.

The advantage of working in the FS-regime:
according to equations~\cref{eq:TBMT_main,eq:TBMT_MDM,eq:TBMT_EDM}, the MDM and EDM
angular velocity vectors are orthogonal, meaning that htey add in squares in the net frequency,
and hence the frequency shift associated with the EDM becomes a second-order effect:~\cite[p.~5]{Mane:SpinWheel}
\[
\w \propto \sqrt{\W_{MDM}^2 + \W_{EDM}^2} \approx \W_{MDM} + \frac{\W_{EDM}^2}{2\W_{MDM}}.
\]
This circumstance significantly diminishes the experimental sensitivity.

However, by freezing the particle's spin in the horizontal plane, the only remaining
MDM angular velocity component is aligned with the EDM component, and hence adds to it
linearly, which greatly improves the sensitivity.

\subsection{Realization of the FS condition in a storage ring}\label{sec:FS_in_a_ring}
Storage rings can be classified into three groups:
\begin{enumerate}
	\item Purely magnetic (COSY, NICA, etc),
	\item purely electrostatic (Brookhaven AGS Analog Ring),
	\item combined.
\end{enumerate}

In view of equation~\eqref{eq:TBMT_MDM}, the FS condition cannot be realized in a purely magnetic ring.

For particles like the proton (whose $G>0$), a purely electrostatic ring can
be used in the FS methodological framework, if the beam has the so-called ``magic''
energy, defined as $\gamma_{mag} = \sqrt{(1+G)/G}$.

For particles whose $G<0$ (deuteron) this is impossible, and one is required to use a combined ring.
To realize the FS condition in a combined ring, a radial electric field is introduced~\cite{BNL:Deuteron2008}:
\begin{equation}\label{eq:FS_Er}
E_r = \frac{GB_yc\beta\gamma^2}{1-G\beta^2\gamma^2}.
\end{equation}
