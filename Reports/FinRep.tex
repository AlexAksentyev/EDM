\documentclass{article}

\usepackage[widepage]{repsty}

\begin{document}

\title{Final report}
\author{Aleksandr Aksentev}
\maketitle

\section{Detector counting rate model}
We assume the following model for the detector counting rate:
\begin{equation}\label{eq:DetCntRt}
	N(t) = N_0(t)\cdot\bkt{1 + P\cdot e^{-\sfrac{t}{\LTd}}\cdot\sin(\omega\cdot t + \phi)},
\end{equation}
where $\LTd$ is the decoherence lifetime, and $N_0(t)$ is the counting rate from the unpolarized cross-section.

Since the beam current can be expressed as a function of time as 
\[
	I(t) \equiv N^b(t)\nu = I_0\cdot e^{\lamb t},
\]
$\lamb$ the beam lifetime, the expected number of particles scattered in the direction of the detector during measurement time $\dtc$ is
\begin{align}
N_0(t) & = p\cdot\int_{-\dtc/2}^{+\dtc/2} I(t+\tau)\td\tau \notag                    \\
& = p\cdot\frac{\nu N_0^b}{\lamb} e^{\lamb t}\cdot \bkt{e^{\lamb\sfrac{\dtc}{2}} - e^{-\lamb\sfrac{\dtc}{2}}} \notag \\
& \approx \underbrace{p\cdot\nu N_0^b e^{\lamb t}}_{\text{rate}~r(t)} \cdot\dtc,
\end{align}
where $p$ is the probability of ``useful'' scattering (approximately 1\%).

The actual number of detected particles will be distributed as a Poisson distribution
\[
	P_{N_0(t)}(\tilde{N}_0) = \frac{\bkt{r(t)\dtc}^{\tilde{N}_0}}{\tilde{N}_0!}\cdot e^{-r(t)\dtc},
\]
hence $\SD{\tilde{N}_0}^2(t) = N_0(t)$. %In the limit of large $N_0(t)$, one can use the Gaussian approximation.

We are interested in the expectation value $N_0(t) = \Xpct{\tilde{N}_0(t)}$, and its variance $\SD{N_0}(t)$. Those are estimated in the usual way,~\cite{CountRateStat} as 
\begin{align*}
	\avg{\tilde{N}_0(t)}[\dtm] &= \frac{1}{\Ncm}\sum_{i=1}^\Ncm \tilde{N}_0(t_i), ~ \Ncm = \dtm/\dtc,
\shortintertext{and} 
	\SD{\tilde{N}_0(t)}[\dtm] &= \frac{1}{\Ncm}\sum_{i=1}^\Ncm \bkt{\tilde{N}_0(t_i) - \avg{\tilde{N}_0(t_i)}[\dtm]}^2.
\end{align*}
($\dtm$ is the event measurement time, $\dtc$ is the polarimetry measurement time.) A sum of random variables, $N_0(t)$ is normally distributed.

The standard error of the mean then is % abuse of notation here (SD in place of SE) for aesthetic reasons
\begin{align*}
\SD{N_0}(t) & = \SD{\tilde{N}_0}(t)/\sqrt{\Ncm} = \sqrt{N_0(t)\frac{\dtc}{\dtm}}            \\
& \approx \sqrt{\frac{p\cdot\nu N_0^b}{\dtm}}\cdot\dtc \cdot\exp\bkt{\frac{\lamb}{2}\cdot t}.
\end{align*}
\newcommand{\A}{\frac{1}{\sqrt{p\cdot\nu N_0^b}}}

Relative error grows:
\begin{equation}\label{eq:MeasRelErr}
	\frac{\SD{N_0}(t)}{N_0(t)} \approx \frac{A}{\sqrt{\dtm}}\cdot\exp\bkt{-\frac{\lamb}{2}t} = \frac{A}{\sqrt{\dtm}}\cdot\exp\bkt{\frac{t}{2\LTb}},~ A=\A.
\end{equation}

\section{Figure of merit}
\newcommand{\Asym}{\mathcal{A}}
A measure of the beam's polarization is the relative asymmetry of detector counting rates:~\cite[p.~17]{Eversmann}
\begin{equation}\label{eq:AsymDef}
	\Asym = \frac{N(\frac\pi2) - N(-\frac\pi2)}{N(\frac\pi2)+N(-\frac\pi2)}.
\end{equation}

In the simulation to follow, the function fitted to the asymmetry data is:
\begin{equation}\label{eq:xFOM}
	\Asym(t) = \Asym(0)\cdot e^{\lamd\cdot t}\cdot\sin\bkt{\omega\cdot t + \phi},
\end{equation}
with three nuisance parameters $\Asym(0)$, $\lamd$, and $\phi$. 

Due to the decreasing beam size, the measurement of the figure of merit is heteroscedastic. From~\cite[p.~18]{Eversmann}, the heteroscedasticity model assumed is
\begin{equation}\label{eq:AsymHtsk}
	\SD{\Asym}^2(t) \approx \frac{1}{2N_0(t)}.
\end{equation}

\section{Conditions for maximum precision}
\DeclareDocumentCommand{\stat}{s}{\IfBooleanTF{#1}{X_{tot}}{\frac{\SD{\meas}^2}{\SE{\hat\omega}^2\cdot \var[w]{t}}}}
\newcommand{\dtnd}{\dt_{zc}}
\newcommand{\SNR}{\text{SNR}}
\newcommand{\deq}{\overset{\triangle}{=}}

Assuming a Gaussian error distribution with mean zero and variance $\SD{\meas}^2$, the maximum likelihood estimator for the variance of the frequency estimate of the cross-section asymmetry $\Asym$ can be expressed as
\begin{align}
\var{\hat\omega} &= \frac{\SD{\meas}^2}{X_{tot}\cdot \var[w]{t}}, \label{eq:VarW}
\shortintertext{with}
X_{tot} &= \sum_{j=1}^{\Nm} x_j = \sum_{s=1}^{\Nnd}\sum_{j=1}^{\Nmnd} x_{js}, \notag\\
\var[w]{t} &= \sum_i w_i \bkt{t_i - \avg{t}[w]}^2,~ \avg{t}[w] = \sum_i w_i t_i, \notag\\
w_i &= \frac{x_i}{\sum_j x_j},~ x_i = (\Asym(0)\exp(\lamd t_i))^2\cos^2(\omega t_i + \phi) = \bkt{\mupp}^2. \notag
\end{align}

In the expression above, $X_{tot}$ is the total Fisher information of the sample, and $\var[w]{t}$ is a measure of its time-spread. It can be observed that by picking appropriate sampling times, one can raise the $X_{tot}$ term, since it is proportional to a sum of the signal's time derivatives. If the oscillation frequency and phase are already known to a reasonable precision, further improvement can be achieved by the application of a sampling scheme in which measurements are taken only during rapid change in the signal (sampling modulation). Improvement here is limited by the polarimetry sampling rate.

Both the $\var[w]{t}$ and $X_{tot}$ terms are bounded as a result of spin tune decoherence. We can express $\sum_{j=1}^{\Nmnd} x_{js} = \Nmnd \cdot x_{0s}$, for some mean value $x_{0s}$ at a given node $s$. $\Nmnd$ is the number of asymmetry measurements per node. The period of time during which measuring takes place, $\dtnd$, is termed \emph{compaction time}. The value of the sum $\sum_{j=1}^{\Nmnd} x_{js}$ falls exponentially due to decoherence, hence $x_{0s} = x_{01}\exp{(\lamd\cdot \frac{(s-1)\cdot\pi}{\omega})}$. Therefore,
\begin{align}
	X_{tot} & = \Nmnd\cdot x_{01} \cdot \frac{\exp{\bkt{\frac{\lamd\pi}{\omega}\Nnd}}-1}{\exp{\bkt{\frac{\lamd\pi}{\omega}}}-1} 
	\equiv \Nmnd \cdot x_{01}\cdot g(\Nnd); \label{eq:FItot}\\
	x_{01}  & = \frac{1}{\dtnd}\int_{-\dtnd/2}^{+\dtnd/2}\cos^2(\omega\cdot t)\td t = \frac12\cdot \bkt{1 + \frac{\sin\omega\dtnd}{\omega\dtnd}},                                    \label{eq:MeanFIZC}   \\
	\Nmnd   & = \frac{\dtnd}{\dtm}. \label{eq:NumMeasNode}
\end{align}

Eq.~\eqref{eq:FItot} provides us with a means to estimating the limits on the duration of the experiment. In Table~\ref{tbl:FItot}, the percentage of the total Fisher information limit, the time in decoherence lifetimes by which it is reached, and the signal-to-noise ratio by that time, are summarized. The signal-to-noise ratios are computed according to:
\begin{equation}\label{eq:TauRatioSNR}
\SNR \deq \frac{\Asym(0)\cdot e^{-\sfrac{t}{\LTd}}}{\SD{\Asym}(t)} 
	\approx \sqrt{2\cdot p\cdot\nu N_0^b\cdot \dtc}\cdot \Asym(0)\cdot \exp\bkt*{-\frac{t}{\LTd}\cdot\bkt{1+\frac12\frac{\LTd}{\LTb}}},
%	 \notag \\
%	&\approx \Asym(0) \exp\bkt*{-\frac{t}{\LTd}\cdot\bkt{1+\frac12\frac{\LTd}{\LTb}}},
\end{equation}
%in which the factor before $\Asym(0)$ is approximately equal to 1 as a result of: $\SD{N_0}(0)/N_0(0)\approx 3\%$ with 2000 polarimetry measurements per asymmetry measurement ($\dtm = 2000\cdot\dtc$).
in which, from $\SD{\Asym(0)}/\Asym(0) \approx 3\%$, the factor before the exponent is 33.
\begin{table}[h]
	\centering
	\caption{Total Fisher information, by what time it is reached,\\ and the corresponding signal-to-noise ratio.\label{tbl:FItot}}
	\begin{tabular}{rrr}
		\hline
		FI limit (\%) & Reached ($\times\LTd$) &  SNR \\ \hline
		           95 &                    3.0 &  0.4 \\
		           90 &                    2.3 &  1.1 \\
		           70 &                    1.2 &  5.5 \\
		           50 &                    0.7 & 11.7 \\ \hline
	\end{tabular}
\end{table}

\subsection{Related formula}
Eq.~\eqref{eq:VarW} can be rewritten in physical terms assuming zero-decoherence ($\lamd=0$) and uniform sampling with sampling period $\dt$:
\begin{align*}
	\stat* &= \sum_{k=1}^K \Asym^2(0)\cos^2(\omega t_k + \phi) = \frac12 \Asym^2(0)\cdot K, \\
	\var[w]{t} &= \sum_{k=1}^K(k\dt - \avg{t}[w])^2\underbrace{w_k}_{1/K} \\
				&\approx \frac{\dt^2}{12}K^2 = \frac{T^2}{12},
\intertext{and so}					
	\var{\hat{\omega}} &= \frac{24}{KT^2}\cdot\bkt{\frac{\SD{\meas}}{\Asym(0)}}^2.
\end{align*}

\section{Simulation}
We simulated data from two detectors with parameters gathered in Table~\ref{tbl:DetCntRtParam} for $T_{tot}=1000$ seconds, sampled uniformly at the rate $f_s = 375$ Hz. These figures are chosen for the following reason: the beam size in a fill is on the order of $10^{11}$ particles; if we want to keep the beam lifetime equal to the decoherence lifetime, we cannot exhaust more than 75\% of it; only 1\% of all scatterings are of the sort we need for polarimetry, so we're left with $\vp{7.5}{8}$ useful scatterings. A measurement of the counting rate $N_0(t)$ with a precision of approximately 3\% requires somewhere in the neighborhood of 2000 detector counts, which further reduces the number of events to $\vp{3.75}{5}= f_s\cdot T_{tot}$. One thousand seconds is the expected duration of a fill, hence $f_s = 375$ Hz. 

Relative measurement error for the detector counting rates is depicted in Figure~\ref{fig:LRDetErr}; the cross-section asymmetry, computed according to Eq.~\eqref{eq:AsymDef}, is shown in Figure~\ref{fig:Asym}.
To these data we fit via Maximum Likelihood a non-linear heteroscedastic model\footnote{R package nlreg.~\cite{NLREG}} given by Eq.~\eqref{eq:xFOM}, with the variance function for the weights given by Eq.~\eqref{eq:AsymHtsk}. The fit results are summarized in Table~\ref{tbl:FitRes}.
\begin{table}[h]
	\centering
	\caption{Detector counting rates' model parameters\label{tbl:DetCntRtParam}}
	\begin{tabular}{cccc}
		\hline
		         &   Left   &     Right     &  \\ \hline
		 $\phi$  & $-\pi/2$ &   $+\pi/2$    &   rad   \\
		$\omega$ &  \multicolumn{2}{c}{3}   & rad/sec \\
		  $P$    & \multicolumn{2}{c}{0.4}  &  \\
		 $\LTd$  & \multicolumn{2}{c}{721}  &   sec   \\
		 $\LTb$  & \multicolumn{2}{c}{721}  &   sec   \\
		$N_0(0)$ & \multicolumn{2}{c}{6730} &  \\ \hline
	\end{tabular}
\end{table}

\begin{figure}[h]
	\centering
	\includegraphics[scale=.85]{img/Final/LR_detector_relErr}
	\caption{Relative counting rate measurement error for the left and right detectors as a function of time.\label{fig:LRDetErr}}
\end{figure}

\begin{figure}[h]
	\centering
	\includegraphics[scale=.85]{img/Final/Asymmetry}
	\caption{Expectation value (red line) and sample measurements (black dots) of the cross-section asymmetry.\label{fig:Asym}}
\end{figure}

\begin{table}[h]
	\centering
	\caption{Fit results\label{tbl:FitRes}}
	\begin{tabular}{crrc}
		\hline
		           & Estimate &             SE &  Unit   \\ \hline
		$\Asym(0)$ &   0.400 & $\vp{9.03}{-5}$ &         \\
		 $\lamd$   &  -0.001 & $\vp{7.86}{-7}$ &  1/sec  \\
		 $\omega$  &   3.000 & $\vp{7.55}{-7}$ & rad/sec \\
		  $\phi$   &  -1.571 & $\vp{2.25}{-2}$ &   rad   \\ \hline
	\end{tabular}
\end{table}

\subsection{Modulation gains}
If our initial frequency estimate obtained from a time-uniform sample has a standard error on the order of $\vp{1}{-6}$ rad/sec, simulation shows the standard error of the estimate can be improved to $\approx \vp{5.8}{-7}$ rad/sec.

\begin{thebibliography}{9}
	\bibitem{CountRateStat}
	\url{http://www.owlnet.rice.edu/~dodds/Files331/stat_notes.pdf}.
	
	\bibitem{Eversmann}
	D. Eversmann et al. ``Analysis of the Spin Coherence Time at the Cooler Synchrotron COSY,'' 2013. \url{http://wwwo.physik.rwth-aachen.de/fileadmin/user_upload/www_physik/Institute/Inst_3B/Mitarbeiter/Joerg_Pretz/DEMasterarbeit.pdf}.
	
	\bibitem{NLREG}
	\url{https://cran.r-project.org/web/packages/nlreg/index.html}
	
\end{thebibliography}

\end{document}