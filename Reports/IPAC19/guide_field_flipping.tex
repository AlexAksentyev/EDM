\documentclass[a4paper]{jacow}

\usepackage{mathtools}
\usepackage{amsmath}
\usepackage{xfrac}
\usepackage{xparse}
\usepackage{subcaption}
\usepackage{graphicx}
\usepackage{url}
\usepackage{paralist}

\let\oldvec\vec
\renewcommand{\vec}{\boldsymbol}
\DeclareDocumentCommand{\bkt}{sm}{\IfBooleanTF{#1}{\left[ #2 \right]}{\left(#2\right)}}
\DeclareDocumentCommand{\ddt}{m}{\frac{\mathrm{d} {#1}}{\mathrm{d} t}}
\DeclareDocumentCommand{\pddx}{mO{t}O{}}{\frac{\partial^{#3} {#1}}{\partial {#2}^{#3}}}
\newcommand{\w}{\omega}
\newcommand{\W}{\Omega}
\newcommand{\avg}[1]{\langle {#1} \rangle}

\begin{document}
\title{Simulation of the Guide Field Flipping Procedure for the Frequency Domain Method}
\author{A. E. Aksentyev$^1$\footnotemark[2], Institut f\"ur Kernphysik, Forschungszentrum J\"ulich, J\"ulich, Germany \\$^1$ also at National Research Nuclear University ``MEPhI,'' Moscow, Russia}
%\author{$^1$ also at National Research Nuclear University ``MEPhI,'' Moscow, Russia}
%\ead{a.aksentev@fz-juelich.de}
\maketitle
\footnotetext[2]{a.aksentev@fz-juelich.de}
\begin{abstract}
  The spin vector of a particle injected into a perfectly aligned storage ring precesses about the vertically-orientated guide field. In the presence of an Electric Dipole Moment (EDM), the spin precession axis acquires a proportional radial component.
  However, in an imperfect ring, rotational magnet misalignments induce a radial component to the spin precession axis, related to the Magnetic Dipole Moment (MDM). In the Frequency Domain Method, this additional precession is dealt with by consecutively injecting the beam in opposite directions, and constructing the EDM estimator as the sum of the clockwise and counter-clockwise vertical plane precession frequencies. Since the radial MDM component changes sign when the magnetic field direction is reversed, it cancels in the sum, leaving only the EDM effect. 
  In order to reproduce the guide field magnitude with a precision sufficient for the cancellation of the MDM effect, we propose to calibrate the guide field via the horizontal plane precession frequency. In the present work we describe the algorithm of the field flipping procedure, and make a numerical simulation.
\end{abstract}

\section{Spin dynamics in a storage ring}
The dynamics of a spin-vector $\vec s$ in a magnetic field $\vec B$ and an electrostatic field $\vec E$ is described by the Thomas-BMT equation. Its generalized version, accounting for the effect of the particle's electric dipole moment, can be written in the rest frame as:
\begin{subequations}
  \begin{align}
    \ddt{\vec s} &= \vec s\times \bkt{\vec\W_{MDM} +\vec\W_{EDM}}, \label{eq:TBMT_main}
    \intertext{where the magnetic (MDM) and electric (EDM) dipole moment angular velocities $\vec\W_{MDM}$ and $\vec\W_{EDM}$ }
    \vec\W_{MDM} &= \frac qm \bkt*{G\vec B - \bkt{G - \frac{1}{\gamma^2-1}}\frac{\vec E\times\vec\beta}{c}},\label{eq:TBMT_MDM} \\
    \vec\W_{EDM} &= \frac qm \frac\eta2 \bkt*{\frac{\vec E}c + \vec\beta\times \vec B}.\label{eq:TBMT_EDM}
  \end{align}
\end{subequations}
In the above equations, $m,~q,~G=(g-2)/2$ are, respectively, the mass, charge, and magnetic anomaly of the particle; $\beta = \sfrac{v_0}{c}$ is its normalized speed; $\gamma$ its Lorentz-factor. The EDM factor $\eta$ is defined by the equation $d = \eta\frac{q}{2mc}$, in which $d$ is the particle EDM, $s$ its spin.

%% In the standard formalizm one operates with the spin transfer matrix~\cite[p.~4]{COSY:SpinTuneMapping}
%% \[
%% \boldsymbol{t}_R = \exp\bkt{-i\pi\nu_s\vec\sigma\cdot\bar n} = \cos\pi\nu_s - i (\vec\sigma\cdot\bar n)\sin\pi\nu_s,
%% \]
%% where $\nu_s = \sfrac{\W_s}{\W_{cyc}}$, the ratio of the particle's spin precession frequency to its cyclotron frequency, is termed \emph{spin tune}, and $\bar n$, termed the \emph{invariant spin axis}, defines the spin precession axis. They relate to the spin precession angular velocity as in $\vec\W_s = \w_{cyc}\cdot \nu_s\bar n$.

\section{BNL Frozen Spin method}
The original method for the measurement of the electric dipole moment of an elementary particle was first proposed by the Storage Ring EDM Collaboration~\cite{BNL:SREDM} of the Brookhaven National Laboratory. In the proposed method,~\cite{BNL:Deuteron2008} a longitudinally-polarized beam is injected into a storage ring designed on the basis of the Frozen Spin (FS) concept: by applying a radial electric field $E_r = \frac{GB_yc\beta\gamma^2}{1-G\beta^2\gamma^2}$,~\cite[p.~10]{BNL:Deuteron2008} the MDM component in~\eqref{eq:TBMT_main} is set to zero: $\vec\W_{MDM} = \vec 0$. Then, any tilting of the beam polarization vector out of the horizontal plane is attributed to the presence of an EDM; specifically, the vertical component $P_y$ grows as
\[
P_y =  P\frac{\W_{edm}}{\W}\sin\bkt{\W t + \Theta_0} \approx P\W_{EDM}\cdot t,
\]
where $\W = \sqrt{\W_{EDM}^2 + \W_{MDM}^2}$.~\cite[p.~8]{BNL:Deuteron2008}

This method has two inherent weaknesses, due to the smallness of the hypothesized EDM value:
\begin{inparaenum}[\itshape a\upshape)]
\item the expected polarization tilt angle after 1,000 seconds is on the order of microradians,~\cite[p.~18]{BNL:Deuteron2008} which is a difficult task for polarimetry~\cite[p.~6]{Mane:SpinWheel} and
\item the main systematic effect, $\W_{syst} \approx \frac{\mu\avg{E_y}}{c\beta\gamma^2}$ ($\mu$ being the MDM of the particle)~\cite[p.~10]{BNL:Deuteron2008} must be reduced to less than $\W_{EDM}$ if one is to measure the polarization tilt angle.
\end{inparaenum}
The considered systematic error is caused by accelerator element misalignment error. For a practical value of 100$\mu$m of element installation error, this means a $\W_{syst}$ on the order of 50 rad/sec.~\cite{Senichev:FDM}

Both these problems can be mitigated if the net precession \emph{frequency}, instead of \emph{phase}, is used as the EDM observable.

\section{Frequency Domain Method}
The Frequency Domain Methodology (FDM)~\cite{Senichev:FDM} was designed specifically to address the problem with element misalignment error. The FS condition condition is fulfilled as in the BNL method; however, instead of the polarization tilt angle, the combined MDM+EDM precession frequency is measured in two cases: once when the beam is injected clockwise, and once counter-clockwise. The EDM-effect is extracted by comparing the measured frequencies. When the guide field polarity is reversed $\vec B \mapsto -\vec B$, $\vec\beta \mapsto -\vec\beta$, and $\vec E \mapsto \vec E$, the precession frequency components change thus:
\begin{subequations}
  \begin{align}
    \W_x^{CW/CCW} &= \W_x^{MDM, CW/CCW} + \W_x^{EDM, CW/CCW}, \\
    \W_x^{MDM, CW} &= -\W_x^{MDM, CCW} \equiv \W_x^{MDM}, \label{eq:FDM_CW_CCW_MDM} \\
    \W_x^{EDM, CW} &= \W_x^{EDM, CCW} \equiv \W_x^{EDM},
    \intertext{and the EDM estimator}
    \hat\W_x^{EDM} &:= \frac12\bkt{\W_x^{CW} + \W_x^{CCW}} \label{eq:FDM_estimator} \\
    &= \W_x^{EDM} + \underbrace{\frac12\bkt{\W_x^{MDM, CW} + \W_x^{MDM, CCW}}}_{\varepsilon \to 0}.
  \end{align}
\end{subequations}

In order to guarantee that $\varepsilon$ is less than the required EDM measurement precision, i.e. that the equation~\eqref{eq:FDM_CW_CCW_MDM} holds with sufficient accuracy, a guide field flipping algorithm has been devised, that uses the horizontal plane precession frequency as a means to calibrating the guide field. 

\section{Calibration algorithm}
The main concern in flipping the direction of the guide field is to accurately reproduce the radial component of teh MDM spin precession frequency due to element misalignment.

The spin precession frequency components are defined by the following equation:~\cite[p.~4]{COSY:SpinTuneMapping}
\[
(\W_x, \W_y, \W_z) = 2\pi\cdot f_{rev} \cdot \nu_s \cdot \bar n,
\]
where $f_{rev}$ is the particle cyclotron frequency, $\nu_s$ and $\bar n$ are its spin tune and invariant spin axis respectively.

We consider the case of a lattice utilizing sextupole elements in order to reduce spin tune dispersion.~\cite{Aksentev:DecohIPAC19} In such a lattice, both $\nu_s$ and $\bar n$ are uniform over a large area of the transverse cross-section of the vacuum tube (and also momentum deviation), and therefore we will assume that all the beam particles share the spin tune and precession axis defined on the reference (closed) orbit.

The reference orbit $\nu_s^{CO}$ is determined by the orbit length, and $\bar n^{CO}$ is determined by both the orbit length and electromagnetic field orientation. When flipping the guide field, the field orientation is preserved, and so the problem of reproducing the MDM angular velocity component comes down to reproducing $\nu_s^{CO}$. This is done directly via reproducing $\W_y^{MDM}$:
in the initial state, $\W_x\gg \W_y, \W_z$ and $\bar n^{CO}\approx \hat x$. Using a spin-rotator (Wien filter), we 
suppress precession about $\hat x$; simultaneously we offset the beam energy from the ``frozen'' value (this is done so as to avoid the unstable ``3D Frozen Spin'' state).  When changing the beam energy, we also adjust the guide field strength accordingly, so as to preserve the reference orbit. Horizontal precession becomes dominant, and $\bar n^{CO}\approx \hat y$. After reversing the guide field polarity, we adjust its strength once again to fulfill the FS condition. Then, after turning off the spin-rotator, and brining the beam energy back to its original level, we get $\bar n^{CO}\approx -\hat x$, $\nu_s^{CCW} = \nu_s^{CW}$, i.e., MDM precession occurs at the same rate as before, but in the opposite direction.

One known systematic error in this procedure is the vertical component $\W_y^{EDM} \propto \vec\beta\times\vec B_x$, where $B_x$ is the misalignment error field, but this error is by default less than the estimated radial EDM frequency component, and can be neglected.

\section{Simulation}

\section{Conclusions}


\begin{thebibliography}{9}

\bibitem{BNL:SREDM}
  M Bai et al., SREDM Collaboration website: \url{https://www.bnl.gov/edm/}

\bibitem{BNL:Deuteron2008}
  D. Anastassopoulos et at., (srEDM Collaboration), ``Search for a permanent electric dipole moment of the deuteron nucleus at the $10^{-29}~e\cdot cm$ level,'' proposal as submitted to the BNL PAC, April 2008.

\bibitem{Mane:SpinWheel}
  S Mane, ``A distillation of Koop's idea of the Spin Wheel.'' arXiv:1509.01167 [physics] \url{http://arxiv.org/abs/1509.01167}
  
\bibitem{Senichev:FDM}
  Y. Senichev, A. Aksentev, A. Ivanov, E. Valetov, ``Frequency domain method of the search for the deuteron electric dipole moment in a storage ring with imperfections.'' In review.

\bibitem{COSY:SpinTuneMapping}
  A. Saleev et al., (JEDI Collaboration), ``Spin tune mapping as a novel tool to probe the spin dynamics in storage rings.'' Phys.Rev.Accel.Beams 20 (2017) no.7, 072801

\bibitem{Aksentev:DecohIPAC19}
  A Aksentev, Y Senichev, ``Spin decoherence in the Frequency Domain Method for the search of a particle EDM.'' Proc. of IPAC19 (2019).
  
%% \bibitem{Senichev:IPAC13}
%%   Y. Senichev, R. Maier, D. Zyuzin, N. Kulabukhova, (JEDI Collaboration), ``Spin tune decoherence effects in electro- and magnetostatic structures.'' Proc. of IPAC13 (2013).

%% \bibitem{COSY:Website}
%%   M. Berz, Kyoko Makino, COSY Infinity website: \url{cosyinfinity.org}


  
\end{thebibliography}
\end{document}
