\documentclass{article}
\usepackage[widepage]{repsty}

\begin{document}

\title{Study of spin-orbit motion in the Frozen and Quasi-Frozen rings}
\maketitle

\section{Spin decoherence effect in a perfectly aligned storage ring}

We study decoherence for two major reasons:
\begin{enumerate}
\item it reduces the measurement time frame;
\item vertical plane decoherence is a source of systematic error.~\cite[p. 8]{Senichev:StorageRingMethod}
\end{enumerate}

Here we consider the decoherence resulting from the beam particles' spin tune difference.

The spin dynamics in a storage ring is governed by the FT-BMT equation,~\cite[p. 4]{JEDI:SpinTuneMapping} In the standard spinor formalism, spin evolution is described by a rotation matrix, whose eigenvector is the spin precession axis, $\bar n$, and eigenvalue the spin tune, $\nu_s$. For particles traveling along the design orbit of a perfectly aligned, flat storage ring, the equilibrium direction of $\bar n$ is vertical.~\cite[p. 1362]{DESY:SpinTune} The spin tune expressions in the electric and magnetic fields are:~\cite[p. 8]{Senichev:StorageRingMethod}
\begin{align}
  \nu_s^E &= \bkt{\frac1{\gamma^2-1}-G}\gamma\beta^2, \\
  \nu_s^B &= \gamma G,
\end{align}
where $G = \frac{g-2}{2}$ is the magnetic moment anomaly, $\gamma = (1-\beta^2)^{-\sfrac12}$ is the particle's Lorentz factor. 

The phase stability principle requires that orbit lengthening is accompanied by a corresponding equilibrium energy shift~\cite{Senichev:Decoh}. This means that particles traveling on different orbits will have differring spin tunes, hence decoherence. Orbit lengthening is due to two reasons:
\begin{inparaenum}[a)]
\item betatron motion,
\item momentum deviation.
\end{inparaenum}

\section{Decoherence suppression via sextupole fields}
Sextupole fields influence orbit lengthening via two paths:
\begin{enumerate}
\item by affecting the momentum compaction factor: $\Delta\alpha_1 = -\frac{S\cdot D_0^3}{L}$
\item by directly changing the orbit length: $\frac{\Delta L}{L} = \mp \frac{S\cdot D_0\cdot \beta_{x,y}\varepsilon_{x,y}}{L}$.
\end{enumerate}

For this reason, the orbit-lengthening decoherence is suppressed by use of three families of sextupoles, put into the maxima of $\beta_x,~\beta_y,~D$ functions respectively.

\section{Simulation}
In the simulation we tried to optimize the strengths of sextupoles placed in the maxima of the $\beta_x,~\beta_y,~D$ functions in order to suppress spin tune decoherence. Optimization was carried out 


\begin{thebibliography}{99}
\bibitem{JEDI:SpinTuneMapping}
  Saleev A, Nikolaev NN, Rathmann F, Augustyniak W, Bagdasarian Z, Bai M, et al. Spin tune mapping as a novel tool to probe the spin dynamics in storage rings. Physical Review Accelerators and Beams [Internet]. 2017 Jul 7 [cited 2018 Oct 8];20(7). Available from: \url{http://arxiv.org/abs/1703.01295}
  
\bibitem{DESY:SpinTune}
  Barber D P, Vogt M, Hoffst\"atter G H. The Amplitude Dependent Spin Tune and the Invariant Spin Field in High Energy Proton Accelerators. \url{http://accelconf.web.cern.ch/AccelConf/e98/PAPERS/THP35G.PDF}

\bibitem{Senichev:StorageRingMethod}
  Yurij Senichev. Search for the Charged Particle Electric Dipole Moments in Storage Rings. In: 25th Russian Particle Accelerator Conference (RuPAC’16), St Petersburg, Russia, November 21-25, 2016 [Internet]. JACOW, Geneva, Switzerland; 2017 [cited 2017 Apr 5]. p. 6–10. Available from: \url{http://accelconf.web.cern.ch/AccelConf/rupac2016/papers/mozmh03.pdf}

\bibitem{Senichev:Decoh}
  Senichev Y, Zyuzin D. SPIN TUNE DECOHERENCE EFFECTS IN ELECTRO- AND  MAGNETOSTATIC STRUCTURES. In: Beam Dynamics and Electromagnetic Fields [Internet]. Changhai, China: JACoW; 2013 [cited 2017 Jul 31]. p. 2579--2581. Available from: \url{https://accelconf.web.cern.ch/accelconf/IPAC2013/papers/wepea036.pdf}
\end{thebibliography}

\end{document}

