\documentclass{article}

\usepackage{phdstyle}
\usepackage[numbers]{natbib}

\newcommand{\SUPPRESSOR}{WIEN FILTER}

\begin{document}

Problem statement. Calibration of the magnetic field by the horizontal precession frequency. Simulation results.
\newpage

Rotational magnet misalignments cause a faking MDM precession in the vertical plane (see section~\ref{sec:SystErr:FakeSignalSim}), which, in the method considered in this work, is dealt with by injecting the beam consecutively in opposite directions and constructing the EDM effect estimator as a sum of the CW and CCW vertical plane precession frequencies (see section~\ref{sec:EDMExp:FDM}, eq~\eqref{eq:FDM_estimator}).

Since the precession frequencies are determined by spin tune and the invariant spin axis via $(\W_x, \W_y, \W_z) = 2\pi\cdot f_{rev} \cdot \nu_s \cdot \bar n$, it is sufficient to equalize the $\gamma_{eff}$ of the CW and CCW beams in order to ensure the same faking MDM magnitude. This is because:
\begin{inparaenum}[\itshape a\upshape)]
\item the reference orbit spin precession axis preserves orientation and changes sign from $\bar n \mapsto -\bar n$ when $\vec B \mapsto -\vec B$;
\item spin tune $\nu_s$ is completely determined by $\gamma_{eff}$ (see section~\ref{sec:EffectiveGamma}).
\end{inparaenum}

The calibration of $\gamma_{eff}$ is done directly via the equalization of the horizontal spin precession frequencies. In the initial state the beam is revolving clockwise, $\W_x \gg \W_y, \W_z$, and $\bar n \approx \hat x$. By means of a \SUPPRESSOR, the radial precession is first suppressed. Horizontal precession then becomes dominant, i.e., $\bar n \approx \hat y$. After the reversion of the guide field polarity, the FS condition is again fulfilled. Since the element tilts remain constant during this time, when the \SUPPRESSOR~is turned off again, $\bar n \approx -\hat x$.



\end{document}
