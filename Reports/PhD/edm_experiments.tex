\documentclass{article}

\usepackage{phdstyle}

\let\oldvec\vec
\renewcommand{\vec}{\boldsymbol}

\begin{document}
\tableofcontents
\newpage

\section{Thomas-BMT equation}
The Thomas-BMT equation describes the dynamics of spin vector $\vec s$ in magnetic field $\vec B$ and electrostatic field $\vec E$. Generalized to account for the EDM effects, it can be written (in the rest frame) as follows:~\cite[p. 6]{Eremey:Thesis}
\begin{align}
  \ddt{\vec s} &= \vec s\times \bkt{\vec\W_{MDM} +\vec\W_{EDM}}, \label{eq:TBMT_main}
  \intertext{where the MDM and EDM angular frequencies $\vec\W_{MDM}$ and $\vec\W_{EDM}$ are}
  \vec\W_{MDM} &= \frac qm \bkt*{G\vec B - \bkt{G - \frac{1}{\gamma^2-1}}\frac{\vec E\times\vec\beta}{c}},\label{eq:TBMT_MDM} \\
  \vec\W_{EDM} &= \frac qm \frac\eta2 \bkt*{\frac{\vec E}c + \vec\beta\times \vec B}.\label{eq:TBMT_EDM}
\end{align}
In the equations above, $m,~q,~G$ are the particle mass, electric charge, and anomalous MDM respectively; $\beta = \sfrac{v_0}{c}$, is the ratio of the particle velocity to the speed of light; $\gamma$ is the Lorentz factor. The EDM factor $\eta$ is defined by $d = \eta\frac{q}{2mc}s$, where $d$ is the particle EDM and $s$ is its spin.

\section{Spin tune}
In the standard spinor formalism, the spin transfer matrix per turn in a ring R equals~[p. 4]\cite{COSY:SpinTuneMapping}
\begin{equation}
  \boldsymbol t_R = \exp\bkt{-i\pi\nu_s\vec\sigma\cdot\vec c} = \cos\pi\nu_s - i(\vec\sigma\cdot\vec c)\sin\pi\nu_s,
\end{equation}
where $\vec\sigma$ is the Pauli matrix vector, $\vec c$ is a unit vector, pointing along the local spin precession axis. The spin precession angular velocity can be written as
\[
\vec\W_s = 2\pi f_s\vec c = 2\pi f_R\nu_s\vec c,
\]
where $f_R$ is the beam revolution frequency, and $\nu_s$ is the \emph{spin tune}, i.e. the number of spin revolutions per turn.

\section{The Frozen Spin concept}
It can be observed in eq~\eqref{eq:TBMT_MDM} that, in the absence of an EDM, the spin of a beam particle can be frozen along its momentum direction: $\vec\W_{MDM}=\vec 0$, i.e., the so-called Frozen Spin (FS) condition can be realized.

EXPLAIN HOW FS IS A RESONANCE.~\cite{COSY:ImperfectionResonance}

EDM experiment methodologies can be classified into:
\begin{inparaenum}[1)]
\item Resonance; those utilizing the FS condition to maximize the EDM signal, and
\item Non-resonance.
\end{inparaenum}

\subsection{Storage rings for EDM search experiments}
Storage rings can be classified into three groups:
\begin{enumerate}
\item purely magnetic (like COSY, NICA, etc),
\item purely electrostatic (Brookhaven AGS Analog Ring),
\item combined rings.
\end{enumerate}

In view of eq~\eqref{eq:TBMT_MDM}, the FS condition cannot be realized in a purely magnetic ring. For a number of particles, such as the proton, whose $G>0$, a purely electrostatic ring can be used in a resonance-type EDM experiment methodology. For particles with $G<0$ (such as the deuteron), this is not an option, and a combined ring must be used. In order to realize the FS condition in a combined ring, a radial E-field $E_r = \frac{GBc\beta\gamma^2}{1-G\beta^2\gamma^2}$ is introduced.~\cite{BNL:Deuteron2008} 

\section{Non-resonance methods}

\subsection{COSY Spin Tune Mapping + RF Wien Filter Method}

\section{Resonance methods}

\subsection{BNL Frozen Spin Method}

\subsection{Koop's Spin Wheel Modification}

\subsection{Frequency Domain Method}

\begin{thebibliography}{99}
\bibitem{Eremey:Thesis}
  Eremey Valetov. FIELD MODELING, SYMPLECTIC TRACKING, AND SPIN DECOHERENCE FOR EDM AND MUON G-2 LATTICES [Internet]. [Michigan, USA]: Michigan State University; Available from: \url{http://collaborations.fz-juelich.de/ikp/jedi/public_files/theses/valetovphd.pdf}
\bibitem{COSY:SpinTuneMapping}
  Saleev A, Nikolaev NN, Rathmann F, Augustyniak W, Bagdasarian Z, Bai M, et al. Spin tune mapping as a novel tool to probe the spin dynamics in storage rings. Physical Review Accelerators and Beams [Internet]. 2017 Jul 7 [cited 2018 Oct 8];20(7). Available from: \url{http://arxiv.org/abs/1703.01295}
\bibitem{BNL:Deuteron2008}
  D. Anastassopoulos, V. Anastassopoulos, D. Babusci. AGS Proposal: Search for a permanent electric dipole moment of the deuteron nucleus at the 10 −29 e · cm level. [Internet]. BNL; 2008 [cited 2016 Nov 25]. Available from: \url{https://www.bnl.gov/edm/files/pdf/deuteron_proposal_080423_final.pdf}
\bibitem{COSY:ImperfectionResonance}
  Stockhorst H. Polarized Proton and Deuteron Beams at COSY. arXiv:physics/0411148 [Internet]. 2004 Nov 16 [cited 2018 Oct 22]; Available from: \url{http://arxiv.org/abs/physics/0411148}


\end{thebibliography}
\end{document}
