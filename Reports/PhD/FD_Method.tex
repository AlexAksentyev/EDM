\documentclass{article}

\usepackage[widepage]{repsty}

\newcommand{\W}{\Omega}
\newcommand{\w}{\omega}
\newcommand{\MDM}{^\mathrm{MDM}}
\newcommand{\EDM}{^\mathrm{EDM}}
\newcommand{\CW}{^\mathrm{CW}}
\newcommand{\CCW}{^\mathrm{CCW}}



\begin{document}
\title{Frequency domain method for the dEDM search}
The original proposal was to measure the growth of the vertical polarization component $\Delta P_V$ over a prolonged period of time.~\cite{BNL_proposal} This approach, however, has a weakness in that it requires the MDM spin precession in the y-z plane to be neglibible compared to the EDM one. For a sensitivity of $10^{-29} e\cdot cm$, a change in polarization asymmetry of about $5\cdot 10^{-6}$ has to be detectable.~\cite[p. 18]{BNL_proposal} This fact puts a stringent constraint on the element alignment precision. Specifically, it would have to be at least as good as $10^{-10} m$;~\cite{Senichev:alignment} otherwise the weak EDM signal would be drowned by the stronger MDM precession. Such precision, of course, is not feasible.

This was the reason for the development of the Frequency Domain Method (FDM). In this method we give up the attempts on minimizing the parasitic MDM precession about the radial axis, and instead measure the total precession frequency $\vec\W = \vec\W\EDM + \vec\W\MDM$. Then two versions of the experiment have to be performed: one in which $\vec\W = \vec\W\EDM + \vec\W\MDM$, and one in which $\vec\W = \vec\W\EDM - \vec\W\MDM$. This is accomplised by reversing the polarity of the guiding magnetic field, and injecting the beam in the reverse direction (thereinafter referred to as CW \& CCW beams). The required frequency is then estimated from the statistic $\vec\W := \frac12\bkt{\vec\W\CW + \vec\W\CCW}$.

At first glance, there are several problems apparent with this method:
\begin{enumerate}
\item The necessity of double injection for the computation of a single statistic. The problem here is that we cannot guarantee that the two beams follow the same orbit, and hence experience the same fields. This is problematic for two reasons \label{itm:Injection}
  \begin{inparaenum}[a)]
  \item a spin rotation is proportional to the experienced field strength, and
  \item spin rotations are not commutative.
  \end{inparaenum}
  
  \item The necessity of reversing the polarity of the guiding field. This is essentially the same problem (due to a different cause) as above: the reversed magnetic field must have the same absolute value as the direct one, else $\W\MDM$ is not reproduced. \label{itm:Polarity}
\end{enumerate}

In what follows we argue that double injection does not pose a problem, so long as 
\begin{inparaenum}[a)]
\item the frozen spin condition is strictly observed (rotations only occur in one plane), and
\item the parameter $\gamma_{eff}(\gamma_s, x_0, y_0, \sfrac{\Delta p_0}{p_s})$ is kept constant.
\end{inparaenum}
We will also provide a methodology for ensuring that the MDM precession frequency in the y-z plane is reproduced with sufficient precision.

\section{Effective gamma}
Let us consider the injection space $(x,y,d)$, where $x$, $y$, are the initial offsets of the particle from the reference orbit, and $d := \frac{\Delta\gamma}{\gamma_0}$ is the energy deviation. A particle injected into the storage ring can then be represented by a point in injection space. Each point in the injection space corresponds to an orbit traveled by the particle. We introduce the concept of gamma effective as an equivalence relation on injection space: if two points in injection space are characterized by the same spin tune, they are said to possess the same gamma effective. 

The origin of the notion can be traced to decoherence studies in the storage ring for the dEDM experiment,~\cite{Senichev:StorageRingMethod} where it is instrumental in understanding the mechanism by which sextupole fields can be used in order to increse the spin coherence time. Here, we will sketch an outline of the argument presented in~\cite{SenichevStorageRingMethod}.

We start with the expressions for the spin tune in the E-,B-fileds:
\begin{equation}\label{eq:spin_tune}
  \begin{cases}
    \nu^B(\gamma) = \gamma B,\\
    \nu^E(\gamma) = \bkt{\frac{1}{\gamma^2-1}-G}\gamma\beta^2,
  \end{cases}
\end{equation}
the Taylor-expansions of which around the synchronous energy point $\gamma_0$ are:
\begin{equation}
  \begin{cases}
    \nu^B(\Delta\gamma) = \Delta\gamma B,\\
    \nu^E(\delta\gamma) = \Delta\gamma\bkt*{-G - \frac{1+G}{\gamma_0^2}} + \Delta\gamma^2\frac{1+G}{\gamma_0^3} + \dots
  \end{cases}
\end{equation}
Here, the gamma deviation of a particle betatron-oscillating about the reference orbit can be expressed as
\begin{equation}\label{eq:delta_gamma}
  \Delta\gamma(t) = \cos\Omega_s t - \frac{\beta^2}{\eta}\bkt*{\bkt{\alpha_1 - \frac{\eta}{\gamma_s^2}}\frac{\Delta\gamma_m^2}{\beta_s^2} + \gamma^2\bkt{\frac{\Delta L}{L}}_\beta},
\end{equation}
where $\frac{\Delta L}{L}$ is the orbit lengthening due to betatron motion, and $\Delta\gamma_m$ is the synchrotron oscillation amplitude. Use of an RF field averages out the cosine term; the remaining, time-independent quantity is called ``gamma effective.''

Use of sextupole fields affects the spin coherence time by modifying the orbit lengths of particles off the referenced orbit. This modification is such as to guarantee [REFERENCE TO COSY INFINITY SIMULATION] that the spin tune is near constant (and equal that of the reference particle's) in a large volume of injection space.

\section{Change of polarity problem}
This problem has its roots in the inability of our magnetic field measuring devices to produce measurements with required precision. This problem, however, can be circumvented in this way:
\begin{enumerate}
\item suppose there's no EDM precession about the y-axis;
\item $\W_y\MDM\propto B_y$, $W_x\MDM\propto B_x$, and $\frac{B_x}{B_y} = \tan\theta$, where $\theta$ is the rotation angle of the element about the optic axis $s$ (tilt) due to misalignment;
\item when we reverse the direction of the field, the relationship between $B_y$ and $B_x$ is preserved;
\item therefore, if $\W_y\MDM\CCW = -\W_y\MDM\CW$, then $B_y\CCW = -B_y\CW$, then $B_x\CCW = -B_x\CW$, and therefore, $\W_x\MDM\CCW = -\W_x\MDM\CW$.
\end{enumerate}

Now instead of the magnetic field, our task is to reproduce the $\W_y\MDM$ frequency, which doesn't per se require our knowledge of the magnetic field strength. The advantage gained is such: while the precision to which the magnetic field has to be reproduced in order to gain the required precision in the reprodution of the frequency is out of our technical capabilities, necessarily, the horizontal precession frequency estimate can be made with a precision not worse than that of the sought-after precession about the radial axis.\footnote{Actually, the vertical precession frequency estimate gets an adequate precision after a year of measurement. So, unless the precision of the horizontal frequency estimate is several orders of magnitude better that the vertical one's, we cannot spend a year of measurement just to switch a polarity once. Which I'm not sure about, though, because $\w_y \gg \w_x$, and we'd have to fit a sine function (or use an FFT), and the precision is always better when fitting a line.}

The question remaining at this point is if the precision to which we can reproduce $\W_y\MDM$ is on par with that of $\W_x\MDM$; that is, if
\begin{equation}\label{eq:calibration_condition}
  |\W_x\MDM\CCW - \W_x\MDM\CW| \leq |\W_y\MDM\CCW - \W_y\MDM\CW|.
\end{equation}

\begin{thebibliography}{99}
\bibitem{BNL_proposal}
  D. Anastassopoulos, V. Anastassopoulos, D. Babusci. AGS Proposal: Search for a permanent electric dipole moment of the deuteron nucleus at the 10 −29 e · cm level. [Internet]. BNL; 2008 [cited 2016 Nov 25]. Available from: \url{https://www.bnl.gov/edm/files/pdf/deuteron_proposal_080423_final.pdf}
\bibitem{Senichev:alignment}
\bibitem{Senichev:StorageRingMethod}
  Yurij Senichev. Search for the Charged Particle Electric Dipole Moments in Storage Rings. In: 25th Russian Particle Accelerator Conf(RuPAC’16), St Petersburg, Russia, November 21-25, 2016 [Internet]. JACOW, Geneva, Switzerland; 2017 [cited 2017 Apr 5]. p. 6–10. Available from: \url{http://accelconf.web.cern.ch/AccelConf/rupac2016/papers/mozmh03.pdf}

\end{thebibliography}
\end{document}
