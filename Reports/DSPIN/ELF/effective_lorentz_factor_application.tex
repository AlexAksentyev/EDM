In order to minimize systematic error $\epsilon$, one needs a way to keep $\W_{MDM}$ constant across multiple runs.

The obvious way of trying to precisely reproduce the guiding field is inefficient for two major reasons:
\begin{enumerate*}[(1)]
	\item standard magnetic field measurement methods do not yield sufficient precision;
	\item the lattice might not be symmetric enough, in terms of spin dynamics, with respect to reversal of the beam circulation direction.
\end{enumerate*}
Hence, we propose a different variable for calibration.

We note that the number of spin revolutions per turn (spin tune $\nu_s$) depends on the particle's  equilibrium-level energy, expressed by the Lorentz factor $\gamma$:
\begin{align}\label{eq:spin_tune_vs_gamma}
\nu_s^B &= G\gamma, \tag{magnetic field}\\
\nu_s^E &= \frac{G+1}{\gamma} - G\gamma. \tag{electric field}
\end{align}

Not all beam particles in a bunch are characterized by the same $\gamma$. A particle involved in betatron
motion will have a longer orbit, and as a direct consequence of the phase stability principle,
in an accelerating structure utilizing an RF cavity, its equilibrium energy level 
must increase.

The effective Lorentz factor is a generalization of the regular Lorentz factor accounting for betatron motion-related orbit lengthening and non-linearity of the momentum compaction factor.

It has been shown in~\cite[p.~56]{Aksentev:Thesis} that a particle's spin tune can be described by a univariate function; we associate the argument of that function with the effective Lorentz factor. Consequently, spin-vectors of two particles characterized by the same value of the effective Lorentz factor precess as the same rate.

Therefore, if the CW and CCW beam centroids' have equal $\geff$, we can expect the MDM components of the spin precession angular velocities to be equal as well.
