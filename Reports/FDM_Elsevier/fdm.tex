\documentclass[12pt]{elsarticle}

\usepackage{amsmath}
\usepackage{xfrac}
\usepackage{xparse}
\usepackage{url}
\usepackage{hyperref}

%% \usepackage[utf8]{inputenc}
%% \usepackage[T1]{fontenc}
%% \usepackage[english,russian]{babel}

\usepackage{tikz-cd}
\usepackage{cleveref}

\usepackage[inline, shortlabels]{enumitem}
\setlist[enumerate,1]{label=\textit{\alph*)}}

\newcommand{\w}{\omega}
\newcommand{\W}{\Omega}
\newcommand{\nbar}{\bar n}
\DeclareDocumentCommand{\ddt}{m}{\frac{\mathrm{d} {#1}}{\mathrm{d} t}}
\DeclareDocumentCommand{\bkt}{sm}{\IfBooleanTF{#1}{\left[ #2 \right]}{\left(#2\right)}}
\newcommand{\D}{\Delta}
\DeclareDocumentCommand{\g}{s}{\gamma\IfBooleanT{#1}{_{eff}}}
\newcommand{\td}{\mathrm{d}}
\newcommand{\avg}[1]{\langle{#1}\rangle}
\newcommand{\wedm}{\w_{edm}}
\newcommand{\wimp}{\w_{\avg{E_v}}}
\newcommand{\wsw}{\w_{SW}}

\newcommand{\vsig}{\vec\sigma}
\newcommand{\tmi}{\Theta^{mi}}
\newcommand{\nbmi}{\nbar_{mi}}
\newcommand{\tsp}{\Theta^+}
\newcommand{\tsm}{\Theta^-}
\newcommand{\nbsol}{\nbar_{sol}}
\newcommand{\hx}{\hat x}
\newcommand{\tauR}{\tau_{ring}}
\newcommand{\tauS}{\tau_{sol}}
\newcommand{\Bsp}{\avg{B_{sol}^+}}
\newcommand{\Bsm}{\avg{B_{sol}^-}}

\newcommand{\Dbeta}{\beta_1 - \beta_2}
\newcommand{\aGz}{\avg{G_z}}

\newdefinition{rmk}{Remark}

\newcommand{\FDM}{Frequency Domain Method} % the mathod should be renamed b/c FD is too unspecific a name

% ADDING LINENUMBERS FOR REVIEWING:
\usepackage{lineno}

\begin{document}
\begin{frontmatter}
\title{\FDM to Search for the Deuteron Electric Dipole Moment in a Storage Ring Environment}

\author[FZJ,INR,MEPHI]{Alexander Aksentev\corref{cor1}}
\ead{alexaksentyev@gmail.com}

\author[INR]{Yury Senichev}
\ead{y.senichev@inr.ru}

\author[MSU]{Eremey Valetov}
\ead{eremey@valetov.com}

\cortext[cor1]{Corresponding author}

\address[FZJ]{Institut f\"ur Kernphysik (IKP-2), Forschungszentrum J\"ulich,
 % 52428 Wilhelm-Johnen-Stra\ss{}e,
  J\"ulich, Germany}
\address[INR]{Institute for Nuclear Research of the Russian Academy of Sciences,
  %117312 60th October Anniversary pr. 7a,
  Moscow, Russia}
\address[MEPHI]{National Research Nuclear University ``MEPhI,''
  %115409 Kashira Highway 31,
  Moscow, Russia}
\address[MSU]{Department of Physics and Astronomy, Michigan State University,
  %220 Trowbridge Rd, East Lansing,
  MI 48824, USA}


\begin{abstract}

\end{abstract}


\maketitle

\begin{keyword}
  electric dipole moment; storage ring; frozen spin; \FDM.
\end{keyword}

\end{frontmatter}

\tableofcontents
\linenumbers

\section{Introduction}
Spin rotations belong to the Spin(3) group, which is isomorphic to SU(2).
\paragraph{Rotations in SU(2)}
Rotation by angle $\psi$ about direction $\nbar$
\[
R_{\nbar}(\psi) = \exp\bkt*{-i\frac\psi2 (\nbar\cdot\vsig)},
\]
where $\vsig$ is the Pauli matrix vector.

\subsection{General spin rotation matrices}
Denote
\begin{itemize}
\item $(\tmi, \nbmi)$ from machine imperfections;
\item $(\tsp, \nbsol)$ for the $+\D$ solenoidal field;
\item $(\tsm, -\nbsol)$ for the $-\D$ solenoidal field.
\end{itemize}

\begin{align}
  R^{+\D} &= \exp\bkt*{-i\bkt{\frac\tmi2(\nbmi\cdot\vsig) + \frac\tsp2(\nbsol\cdot\vsig)}} \notag\\
  &= \exp\bkt*{-\frac i2\bkt{\tmi\nbmi + \tsp\nbsol}\cdot\vsig}, \label{eq:r1}\\
  R^{-\D} &= \exp\bkt*{-\frac i2\bkt{\tmi\nbmi - \tsm\nbsol}\cdot\vsig}, \label{eq:r2}
\end{align}

\section{Preliminary analytic of the Spin Wheel method}
In SW we posit
\begin{equation}\label{eq:SW:master}
  \bkt{\vec\Omega_{MDM}^{+\D}\cdot\hx} = - \bkt{\vec\Omega_{MDM}^{-\D}\cdot\hx}.
\end{equation}

The spin precession angular velocity vector can be expressed via spin tune and invariant spin axis as
\[
\vec\Omega_{spin} = \frac{2\pi}{\tauR}\cdot \nu\cdot\nbar,
\]
hence
\begin{equation}\label{eq:SW:master:spin_tune}
\nu^{+\D}(\nbar_{+\D}\cdot\hx) + \nu^{-\D}(\nbar_{-\D}\cdot\hx) = 0
\end{equation}


From $\D\Theta = \tau\D\Omega$ and $\D\Omega_x^{MDM} = \frac qm G B_x$, and \textbf{assuming}
\begin{equation}\label{eq:SW:assumption:averaging}
B_{sol}^{\pm}\tauS = \avg{B_{sol}^{\pm}}\tauR:
\end{equation}
\begin{equation}\label{eq:sol_spin_tunes}
  \begin{cases}
    \tsp &=  \tauS\frac qm G B_{sol}^+ \stackrel{\eqref{eq:SW:assumption:averaging}}{=} \tauR\frac qm G \Bsp, \\
    \tsm &=  \tauS\frac qm G B_{sol}^- \stackrel{\eqref{eq:SW:assumption:averaging}}{=} \tauR\frac qm G \Bsm.
  \end{cases}
\end{equation}

\begin{rmk}
  Assumption~\eqref{eq:SW:assumption:averaging} is \textbf{required} if we want to obtain
  $B_{sol}^\pm$ from equations of group~\eqref{eq:SW:Koop2015-9}.
\end{rmk}

From eqs~\eqref{eq:r1} and~\eqref{eq:r2}:
\begin{equation}\label{eq:SW:master:spin_tune:expression}
  \begin{cases}
    \tmi\nbmi + \tsp\nbsol = \nu^{+\D}\nbar_{+\D}, \\
    \tmi\nbmi - \tsm\nbsol = \nu^{-\D}\nbar_{-\D}.
  \end{cases}
\end{equation}

Substituting eq~\eqref{eq:SW:master:spin_tune:expression} into~\eqref{eq:SW:master:spin_tune}, and assuming
$\nbsol = \hx$:
\begin{equation}\label{eq:SW:master1}
  2\tmi(\nbmi\cdot\hx) + (\tsp - \tsm) = 0.
\end{equation}

\textbf{Assuming}\footnote{This is a generous assumption implying that $\nbmi = \hx$; i.e., this is \textbf{not}
a non-commutativity-based argument; we assume all spin rotations commute.}
\begin{equation}\label{eq:SW:assumption:commutativity}
\tmi = \tauR\cdot\frac qm G\cdot \avg{B_x}^{mi},
\end{equation}
from~\eqref{eq:SW:master1} and~\eqref{eq:SW:assumption:averaging} obtain:
\begin{equation}\label{eq:SW:master1:assumed}
  2\avg{B_x}^{mi} + \bkt{\Bsp - \Bsm} = 0.
\end{equation}

From eq (9) in Koop2015, assuming in the $+\D$ case the machine imperfections and solenoid fields are co-aligned,
in the $-\D$ anti-aligned:
\begin{equation}\label{eq:SW:Koop2015-9}
  \begin{cases}
    \D^+ &= \frac{\Dbeta}{\aGz}\avg{B_x} = \frac{\Dbeta}{\aGz}\bkt{\avg{B_x}^{mi} + \Bsp}, \\
    &\Rightarrow \Bsp = \frac{\aGz}{\Dbeta}\D^+ - \avg{B_x}^{mi}; \\
    \D^- &= \frac{\Dbeta}{\aGz}\avg{B_x} = \frac{\Dbeta}{\aGz}\bkt{\avg{B_x}^{mi} - \Bsm}, \\
    &\Rightarrow -\Bsm = \frac{\aGz}{\Dbeta}\D^- - \avg{B_x}^{mi}.
  \end{cases}
\end{equation}

Substituting this into~\eqref{eq:SW:master1:assumed}:
\[
2\avg{B_x}^{mi} + \bkt{\frac{\aGz}{\Dbeta}\bkt*{\D^+ - \D^-} - 2\avg{B_x}^{mi}} = 0.
\]

In the original method, we are to make 
\begin{equation}\label{eq:SW:method_condition}
  \D^- = - \D^+,
\end{equation}
so the term in the square brackets is zero, and we are left with
\begin{equation}\label{eq:SW:conclusion}
  \bkt{1 - 1}\avg{B_x}^{mi} = 0.
\end{equation}

So, seems that SW works, but we did two important assumptions here:
\begin{enumerate*}[(a)]
  \item commutativity (in order to get eq~\eqref{eq:SW:assumption:commutativity}), and
  \item ``averaging'' of $B_{sol}$ over the ring (in order to get eq~\eqref{eq:SW:assumption:averaging}
    and remove the $\tauS/\tauR$ from~\eqref{eq:SW:master1:assumed}).
\end{enumerate*}

\begin{rmk}
  If we don't use~\eqref{eq:SW:assumption:commutativity} (but still use~\eqref{eq:SW:assumption:averaging}
  in order to obtain $B_{sol}^\pm$ from group~\eqref{eq:SW:Koop2015-9}), then eq~\eqref{eq:SW:conclusion} becomes
  \begin{equation}\label{eq:SW:conclusion-non-comm}
    \tmi\bkt{\nbmi\cdot\hx} - \frac qmG\cdot\tauR\avg{B_x}^{mi} = 0,
  \end{equation}
  which is not very informative.
\end{rmk}

\begin{rmk}
  To check that eq~\eqref{eq:SW:conclusion-non-comm} is correct, assume~\eqref{eq:SW:assumption:commutativity}.
  Then
  \[
  \frac qmG\tauR\avg{B_x}^{mi}\bkt{\nbmi\cdot\hx} - \frac qmG\tauR\avg{B_x}^{mi}=0,
  \]
  and hence
  \[
  \nbmi\cdot\hx = 1,
  \]
  which is implied by machine imperfection spin rotations adding up commutatively.
\end{rmk}

\begin{rmk}\label{rmk:SW:conclusion}
  In general, since
  \[
  \tmi = \tauR\cdot \frac qmG\sqrt{\avg{B_x^{mi}}^2 + \avg{B_y^{mi}}^2 + \avg{B_z^{mi}}^2},
  \]
  eq~\eqref{eq:SW:conclusion-non-comm} implies that
  \begin{align}\label{eq:nbmi-expression-non-comm}
    (\nbmi\cdot\hx) &= \frac{\frac qmG\tauR\avg{B_x}^{mi}}{\tmi}\notag\\
    &= \frac{\avg{B_x^{mi}}}{\sqrt{\avg{B_x^{mi}}^2 + \avg{B_y^{mi}}^2 + \avg{B_z^{mi}}^2}}.
  \end{align}
  Which is correct.
\end{rmk}

\textbf{Conclusion.} In view of Remark~\ref{rmk:SW:conclusion}, since eq~\eqref{eq:SW:conclusion-non-comm}
implies a valid statement, our conclusion is that the SW method resists the argument from non-commutativity.

\begin{figure}\centering
  \begin{tikzcd}
    \eqref{eq:SW:master1}
    \arrow[r, "\eqref{eq:SW:assumption:commutativity} + \eqref{eq:SW:assumption:averaging}"]
    \arrow[dd, "\eqref{eq:sol_spin_tunes}+" near start, "\eqref{eq:SW:assumption:averaging}+",
      "\eqref{eq:SW:Koop2015-9}" near end]
    & \eqref{eq:SW:master1:assumed} \arrow[r, "\eqref{eq:SW:Koop2015-9}"]
    & \eqref{eq:SW:conclusion}\\
    & & \\
    \eqref{eq:SW:conclusion-non-comm}  
  \end{tikzcd}
  \caption{Argument diagram.}
\end{figure}

%% \paragraph{Argument from averaging} % INVALID
%% Now, the Koop method, as far as we can tell, \textbf{depends} (conceptualy) on our ability
%% to ``average'' the solenoidal field; if
%% we assume that it is a local perturbation --- which is what it actually is, --- then it doesn't contribute
%% to vertical separation of the beams' \emph{closed orbits}. Therefore, under the local perturbation model,
%% the orbit separation is only due to the machine imperfections, and does not change when we vary the
%% solenoidal field.

\section{Assumptions of the Spin Wheel method}
\newcommand{\K}[1]{K(#1)}
\newcommand{\KEz}{K$\avg{E_z}$}
\paragraph{Orbital dynamics}
Koop2015 eq~(7) (henceforth referred to as \K7) and
\begin{equation}\label{eq:Assumptions:KEz}
  \avg{E_z} = \avg{E_z(0)} + \aGz\cdot z \tag{\KEz}
\end{equation}
\begin{align}
  &\rightarrow \avg{z} = \frac{\avg{E_z(0)}}{\aGz} - \frac{\beta}{\aGz}\cdot \avg{B_x} \\
  &\rightarrow \D = \frac{\Dbeta}{\aGz}\avg{B_x}.
\end{align}

This is as far as the argument from the non-linearity of the closed orbit shift dependence
on the magnetic field is concerned.
So long as we believe \K7 and \KEz, we must believe \K9, and hence we cannot use that argument.

\paragraph{Spin dynamics} This is the argument from non-commutativity.
For this argument cf. eq~\eqref{eq:SW:conclusion-non-comm} and Remark~\ref{rmk:SW:conclusion}, and
the following conclusion.

\section{Argumument against the SW method}
The three-fold argument against the SW method is as follows (in the order of strength):
\begin{enumerate}[(1)]
\item The possibility of measuring the vertical orbit separation of two co-circulating beams
  at the sensitivity level of $10^{-12}$~m has not been shown by experiment.
  \textbf{Counter-argument}: there's reference~\cite{Kawal} to commercially-available SQUIDs
  capable of detecting magnetic fields on the order of fT, which is equivalent to the beam separation
  of $10^{-12}$~m. \label{itm:arg1}
\item Even if a SQUID-based BPM is capable of measuring orbit separation to such precision \emph{locally}, the
  evaluation of the \emph{mean} orbit separation requires multiple local measurements,
  and is not identical to the local measurement precision.\label{itm:arg2}
\item Orbital and spin dynamics are idependent of each other, meaning that the observables
  $\vec\Omega$ and $\Delta$ are not directly related.\label{itm:arg2}
\end{enumerate}

Regarding part~\ref{itm:arg1}: a counter to the counter-argument could be that the SQUID
magnetic field measurements aren't linearly related to the beam orbit separation.

Regarding part~\ref{itm:arg2} of the above argument (argument from statistics):
we did a simulation, and confirmed that
\[
\sigma[\avg{\Delta}] = \frac{a_y}{\sqrt{N_{BPM}}},
\]
where $a_y$ is the amplitude of betatron oscillaitons, $N_{BPM}$ is the number of local BPM measurements.


\section{Absence of the $a_y^2$ term in SW equations}

Koop stats from the T-BMT equation \K3, which is a differential equation, defining
\[
\W_x = \frac qmGB_x
\]
locally.
Then, in \K8 he transitions to the average
\[
\avg{\W_x} = \frac qmG\avg{B_x}.
\]
I think this is where he performes
an invalid operation, by just formally including the LHS and RHS into the angle-brackets.

In our formalism,
\begin{equation}\label{eq:spin-tune-main}
\avg{\W_x} \propto G\gamma;
\end{equation}
and since
\[
\gamma \propto \frac{\avg{B_x}}{Q_y} + \kappa\cdot a_y^2,
\]
so is $\avg{\W_x}$.

However, eq~\eqref{eq:spin-tune-main} itself is obtained (AFAIK) from
\begin{align*}
  \W_x &= \frac qmGB_x,\\
  \W_v &= \frac{qB}{m\gamma}, \\
  \frac{\W_x}{\W_v} &= \frac{qGB}{m}\frac{m\gamma}{qB} = \gamma G.
\end{align*}

\begin{thebibliography}{0}
\bibitem{Kawal}
  D. Kawal, ``Relative Beam Position Monitors for the pEDM Experiment.''
  \url{https://apps.fz-juelich.de/pax/paxwiki/images/a/a9/DKawal_longapp_dmk_20110621.pdf}
\end{thebibliography}
\end{document}
