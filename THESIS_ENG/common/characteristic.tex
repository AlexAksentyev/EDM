
One of the main problems of contemporary fundamental physics research is the baryon asymmetry of the universe,
i.e. the prevalence of matter over antimatter in the observed universe. At the present moment there's 
no evidence of the existence of primordial antimatter in our galaxy; the amount of observed antimatter 
is consistent with its production in secondary processes. There's also no detectable background 
gamma radiation that would be expected from nucleon-antinucleon annihilations, if matter and antimatter
galaxies were to coexist in clusters of galaxies.~\cite{Trodden:Baryogenesis} 

In his 1967 paper, Andrei D. Sakharov formulated three necessary conditions that the primordial universe 
must have satisfied for baryogenesis. The discovery of cosmic background radiation and CP-symmetry violation 
in kaon systems~\cite{Fitch:Kaon-CP-violation-1964} motivated the formulation of the conditions. 
The three \emph{Sakharov conditions} are:
\begin{itemize}
	\item violation of the baryon number symmetry;
	\item violation of the discrete C- and CP-symmetries;
	\item a departure from thermal equilibrium.
\end{itemize}


Permanent electric dipole moments (EDMs), if they exist, violate both the P- and T-symmerties, and hence,
via the CPT-theorem, can be linked to CP-symmetry violation.

The Standard Model (SM) of elementary particles allows formalization of CP-invariance viloation
via the Cabibbo-Kobayashi-Masakawa matrix, however the EDM values it predicts for, say, the neuteron,
lie in the range $10^{-33}$ to $10^{-30}$\ecm.~\cite{Harris:Neutron2007}
This implies that particle EDMs can serve as a powerful tool
for discovering physics beyond the SM. For example, CP-violations that are endemic in
supersymmetric teories (SUSY) are such as to span $d_n$ values in the range of
$10^{-29}$ to $10^{-24}~e\cdot$cm.~\cite{JEDI:Website}

The EDM search project started more than 50 years ago. The first neuteron EDM experiment was conducted by
dr. N.F. Ramsey at the end of the 1950s. As a result of that experiment, the upper neuteron EDM bound was set
at $5\cdot 10^{-20}$\ecm.~\cite{Ramsey:Neutron1957} Since then, multiple experiments 
of increasingly high precision were carried out, and at present the upper bound on the neuteron EDM is at
$2.9\cdot 10^{-26}$\ecm.~\cite{Baker:nEDM:Main, Baker:nEDM:Reply}

Up until now, all EDM-searching experiments were performed on electrically-neutral particles, such as
atoms or the neuteron. The idea of searching of a charged particle's EDM in a storage ring environment
appeared during the development of the g-2 experiment~\cite{BNL:g-2:2001} in Brookhaven National Laboratory.

As a result of the BNL experiments, the upper bound on the muon EDM was set at
$10^{-19}$\ecm.~\cite{BNL:muon_ANA:2009} In the 1990s, discussion centered mostly on the
muon EDM experiment~\cite{Farley:SREDM:Muon}, however the deuteron was also considered, in view of its having
a similar magnetic anomaly-to-mass ratio.

In 2004, the Storage Ring EDM Collaboration (srEDM)~\cite{BNL:SREDM} in BNL proposed experiment 970
for detecting the deuteron EDM on the level $10^{-27}$\ecm in a storage ring. Since 2005, a number of feasibility
experiments were run at the KVI cyclotron facility in Groningen to measure broad range spin sensitivities
for deuteron scattering on carbon near 100~MeV.

Experiments at the Cooler Synchrotron COSY (Forschungszentrum J\"ulich, Germany) began in 2008. Later
these experiments developed into a polarized program at COSY with a view of delveloping technologies required
for a storage ring EDM search experiment. In the same year another deuteron EDM experiment was
proposed~\cite{BNL:Deuteron2008}, this time with a sensitivity level $10^{-29}$\ecm in one year of measurement
time.

At the same time it was decided that the proton EDM experiment has a number of technical advantages
ober the deuteron. Among them is the ability to store two counter-circulating beams simultaneously,
which allows the cancellation of T-even systematic effects. Nevertheless, work at COSY continued with
the deuteron because of the investment already made in deuteron operation and the sense that any conclusions would apply to either proton or deuteron beams.~\cite[\textbf{Historical background}]{YellowReport}

In 2011 the JEDI (J\"ulich Elecric Dipoe moment Investigations) collaboration was formed.~\cite{JEDI:Website}
The purpose of the collaboration consists not only in developing technologies for srEDM, but also in
performing a first direct EDM measurement for deuterons.

In 2018, the JEDI collaboration made a first-ever deuteron EDM measurement at COSY. 
Since in a non-Frozen Spin ring
and EDM generates small-amplitude oscillations of the vertical beam polarization component
(at the deuteron momentum 970 MeV/c used in COSY, the oscillation amplitude is on the level $3\cdot10^{-10}$
assuming an EDM $d = 10^{-24}$\ecm), a resonance-type methodology
was used~\cite{COSY:Partially-Frozen-Spin, COSY:SpinTuneMapping}, which uses a custom-design
RF Wien filter~\cite{JSlim:RFWF:Design, JSlim:RFWF:Commisioning} made specially for COSY.

The \textbf{goal} of the present work is to numerically model the 2D Frozen Spin (FS) method 
for searching for the deuteron EDM in a storage ring.

In order to reach this goal, the following \textbf{objectives} were formulated:
\begin{enumerate}
  \item Study spin decoherence in the neigborhood of spin resonance (frozen spin regime) and the sextupole
  method of its suppression. 
  \item Study the effect of betatron oscillations on the validity of the EDM statstic.
  \item Study the effects of lattice optical element misalignments on the systematic error of the EDM statistic.
  \item Model the spin tune calibration procedure used at flipping the polarity of the storage ring guide field
  when counter-injecting the beam.
\end{enumerate}

This research deals with the following problems not analyzed by previous researchers:
\begin{enumerate}[(1)]
  \item Simulation of the spin tune calibration procedure under change of the beam circulation direction. 
  \item Analysis of the effect of betatron oscillations on the EDM statistic.
  \item Systematization of the universally-encountered EDM measurement problems.
  \item Classification of FS-type methods of searching for the EDM of a charged particle in a storage ring.
\end{enumerate}

\textbf{Defended propositions.}
\begin{enumerate}
	\item We confirmed the analytical explanation of the mechanism of the sextupole method for
        suppressing spin decoherence proposed by Y.V. Senichev. % section 2.2.7
  	\item We confirmed the proposed equality of spin tunes of particles having the same value of the
        effective Lorentz factor, and we found an interpretation fo the effective Lorentz factor as
        a measure of the particle's longitudinal emittance. % section 2.5.2
  	\item We showed that the calibration of the vertical MDM precession angular velocity component
        by means of observing spin precession in the horizontal plane is a viable method.
  	\item We proved that perturbations to a particle's spin dynamics due to its betatron motion
        introduce a negligibly small (and also controllable, in the framework of the
        2D FS method) systematic error into the EDM statistic. % pretty well-founded
         % statistics
  	\item We proved that the effective measurement cycle length ranges betwen 2 to 3
        polarization lifetimes.~\footnote{Polarization lifetime here is understood as
        the period of time during which polarizationi drops by a factor of $e$.}
        % this is a fairly well-founded claim
  	\item We showed the possibility of reaching a mean standard error on the EDM on
        the level of $10^{-29}$\ecm in one year of measurement. % if beam 1e11, sampling rate 375 Hz etc
  	\item We proved that the EDM-faking MDM spin precession angular velocity due to machine
        imperfections is independent of the actual distribution of the imperfections~\footnote{
        In the particular imperfection case considered}, and depends only on the expectation value of the
        imperfection distribution. % multiple random distributions
  	\item We proved that at the practical level of element alignment precision non-frequency
        based EDM measurement methodologies cannot be used.
        % b/c spin precession frequency is on the order of 50 rad/sec
\end{enumerate}
