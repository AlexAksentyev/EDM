
\subsection{Уравнение Т-БМТ}\label{sec:TBMT_introduction}
Уравнение Томаса-Баргманна-Мишеля-Телегди описывает динамику спин-вектора $\vec s$ в
магнитном поле $\vec B$ и электростатическом поле $\vec E$. Его
обобщённая версия, включающая влияние ЭДМ, может быть записана (в
системе центра масс пучка) как:~\cite[стр.~6]{Eremey:Thesis}
\begin{subequations}
	\begin{align}
		\ddt{\vec s} &= \vec s\times \bkt{\vec\W_{MDM} +\vec\W_{EDM}}, \label{eq:TBMT_main}
		\intertext{где МДМ и ЭДМ угловые скорости $\vec\W_{MDM}$ и $\vec\W_{EDM}$ }
		\vec\W_{MDM} &= \frac qm \bkt*{G\vec B - \bkt{G - \frac{1}{\gamma^2-1}}\frac{\vec E\times\vec\beta}{c}},\label{eq:TBMT_MDM} \\
		\vec\W_{EDM} &= \frac qm \frac\eta2 \bkt*{\frac{\vec E}c + \vec\beta\times \vec B}.\label{eq:TBMT_EDM}
	\end{align}
\end{subequations}
В уравнениях выше, $m,~q,~G=(g-2)/2$ есть, соответственно, масса, заряд, и
магнитная аномалия частицы; $\beta = \sfrac{v_0}{c}$,
нормализованная скорость частицы; $\gamma$ её Лоренц-фактор. ЭДМ
множитель $\eta$ определяется уравнением $d = \eta\frac{q}{2mc}s$, где
$d$ --- ЭДМ частицы, а $s$ её спин.

В стандартном формализме принято оперировать с матрицей преобразования (поворота) спина за оборот в кольце $R$:~\cite[стр.~4]{COSY:SpinTuneMapping}
\[
\bold{t}_R = \exp\bkt{-i\pi\nu_s\vec\sigma\cdot\bar n} = \cos\pi\nu_s - i (\vec\sigma\cdot\bar n)\sin\pi\nu_s,
\]

где $\nu_s = \sfrac{\W_s}{\W_{cyc}}$ отношение угловой скорости поворота спин-вектора частицы к её циклотронной частоте, называемое \emph{спин-тюн}, а $\bar n$ определяет направление оси прецессии спина, и называется \emph{инвариантной спиновой осью}.

\subsection{Концепция замороженного спина}
Из уравнения~\eqref{eq:TBMT_MDM} можно видеть, что, в отсутствии ЭДМ,
направление вектора спина частицы пучка может быть зафиксировано
относительно её вектора импульса: $\vec\W_{MDM}=\vec 0$; иными словами, можно реализовать
условие замороженности спина (Frozen Spin condition).

Достоинство работы в FS-состоянии в накопительном кольце
следующее: в соответствии с
уравнениями~\cref{eq:TBMT_main,eq:TBMT_MDM,eq:TBMT_EDM}, векторы МДМ и
ЭДМ угловых скоростей ортогональны, а потому в общей скорости
прецессии они складываются квадратично, в связи с чем сдвиг частоты
прецессии, связанный с ЭДМ, становится эффектом второго порядка
величины:~\cite[стр.~5]{Mane:SpinWheel}
\[
\w \propto \sqrt{\W_{MDM}^2 + \W_{EDM}^2} \approx \W_{MDM} + \frac{\W_{EDM}^2}{2\W_{MDM}}.
\]
Это обстоятельство значительно ухудшает чувствительность эксперимента.

Однако, заморозив спин в горизонтальной плоскости, единственная
остающаяся МДМ компонента угловой скорости сонаправлена с ЭДМ
компонентой, а значит складывается с ней линейно. Таким образом,
чувствительность значительно улучшается.

\subsection{Реализация условия замороженности спина в накопительном кольце}\label{sec:FS_in_a_ring}
Накопительные кольца могут быть классифицированы в три группы:
\begin{enumerate}
	\item чисто магнитные (как COSY, NICA, etc),
	\item чисто электростатические (Brookhaven AGS Analog Ring),
	\item комбинированные.
\end{enumerate}

Ввиду уравнения~\eqref{eq:TBMT_MDM}, условие FS не может быть
выполнено в чисто магнитном кольце.

Для некоторого числа частиц, таких как протон, чья $G>0$, чисто
электростатическое кольцо может быть использовано в рамках FS
методологии ЭДМ эксперимента с пучком на так называемой ``магической''
энергии, определяемой как $\gamma_{mag} = \sqrt{(1+G)/G}$.

Для частиц с $G<0$ (таких как дейтрон),это невозможно, и необходимо
использовать комбинированное кольцо. Для того, чтобы реализовать FS
условие в комбинированном кольце, вводится ~\cite{BNL:Deuteron2008} радиальное электрическое
поле величины
\begin{equation}\label{eq:FS_Er}
E_r = \frac{GB_yc\beta\gamma^2}{1-G\beta^2\gamma^2}.
\end{equation}