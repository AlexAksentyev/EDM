
There exist two design approaches to the problem of measuring the deuteron EDM inside a storage ring:
\begin{enumerate*}[\itshape a\upshape)]
	\item the Frozen Spin (FS) lattice, and
	\item the Quasi-frozen spin (QFS) lattice.
\end{enumerate*}

In the following sections we will consider variants of both type lattices.

\subsection{The Frozen Spin lattice} \label{chpt2:lattice:FS_BNL}
In a FS-type lattice, a beam particle's spin vector's horizontal projection and momentum vector
are \emph{continuously} aligned. For the realization of the continuity condition, combined E+B-field
cylindrical spin-rotators placed in the arcs are used. In figure~\ref{fig:BNL_lattice} is shown an example of a
FS-type lattice.~\cite{Senichev:Lattices} This ring is 145.85~m in length and is designed for the deuteron
injection energy 270 MeV. An RF cavity is used in this lattice in order to suppress linear spin decoherence
effects by averaging particle energies. The RF voltge is $V = 100$ kV, RF frequency $f_{RF} = 5\cdot f_{rev}$,
where the cyclotron frequency $f_{rev} = 1.00$ MHz. The remaining non-linear decoherence effects are
suppressed by using three~\footnote{Some authors use two families~\cite{Eremey:Thesis} in this lattice.}
sextupole families.

\begin{figure}[h!]
	\centering
	\includegraphics[width=\linewidth]{images/chapter2/BNL_lattice}
	\caption{A FS lattice variant. Cylindrical E+B spin-rotators are used in the arc sections to fulfill
          the FS condition. (Image is taken from~\cite{Senichev:Lattices}.)\label{fig:BNL_lattice}}
\end{figure}

The main purpose of a FS lattice is to maximize the EDM signature signal. However, it is important to note that,
strictly speaking, the FS condition is held only the reference particle. This is because, as follows from
equation~\eqref{eq:TBMT_MDM}, for any given E- and B-fields there exists a unique value of the Lorentz factor
$\gamma$ at which $\W_y^{MDM} = 0$. Hence, even in a FS lattice, most particles' spin vectors are frozen only
approximately.

\subsection{Quasi-Frozen Spin lattice} \label{chpt2:concept:QFS}
In the QFS design concept, one gives up the continuity property of the FS condition, requiring only that the spin
phase advance (in the rest frame) in the electrostatic ($\Phi_s^E$) and magnetic ($\Phi_s^B$) elements was zero
on average (turn by turn):~\cite{Senichev:Lattices}
\begin{equation*}
	\sum_i \Phi_{s,i}^E = -\sum_j \Phi_{s,j}^B.
\end{equation*}

Following the definition of spin tune (see section~\ref{sec:TBMT_introduction}), a particle's spin vector
placed into an electromagnetic field turns by angle $\Phi_s = \nu_s \cdot \Phi$, where $\Phi$ is
the momentum rotation angle, $\nu_s$ spin tune.

A particle's angular momentum, when placed into a magnetic field $\vec B$ is
\[
\w_B = \frac qm \frac B \gamma,
\]
into an electrostatic $\vec E$:
\[
\w_E = \frac qE \frac{\vec E\times \vec\beta}{c\beta^2\gamma},
\]
from which follow the expressions for spin tune in the electrostatic and magnetic fields:
\begin{equation}
	\begin{cases}
		\nu_s^B &= \gamma G, \\
		\nu_s^E &= \beta^2\gamma\bkt{\frac{1}{\gamma^2-1} - G}.
	\end{cases}
\end{equation}

The QFS lattice design has the advantage of simplicity over the FS one: there's no need to use a combined-field
cylindrical spin rotators; in both QFS lattice variants we consider below are used either
\begin{enumerate*}[\itshape a\upshape)]
\item straignht Wien filters,  or
\item cylindrical electrostatic and magnetic elements separately.
\end{enumerate*}
On the other hand, due to the appearance of a vertical spin precession axis component $\bar n_y$,
the maximum EDM signal amplitude is less compared with the pure FS case. Teh attenuation
factor~\cite{Senichev:QFS_IPAC15}
\[
J_0(\Phi_s) \approx 1 - \frac{\Phi_s^2}{4},
\]
where $\Phi_s$ is the maximum horizontal plane spin phase advance. Assume the phase advance does not
exceed $\pi\cdot \gamma G/2n$; in this context $n$ is the lattice periodicity. Since the deuterin animalous magnetic moment $G = -0.142$, for the QFS lattices considered below $J_0\ge 0.98$.

\subsubsection{QFS lattice design ``6.3''}\label{chpt2:lattice:QFS:6_3}

In Figure~\ref{fig:QFS_6_3_lattice} is presented a QFS design lattice in which the E- and B-fields are
separated in space.~\cite{Senichev:Lattices} Negative radius electrostatic cylindrical deflectors are used
to compensate the spin phase advance related to the MDM precession in the arc sections.~\cite{Senichev:QFS_IPAC15}
The ring is 166.67~m length long and is dessigned for the 270 MeV injection energy. For the suppression
of linear spin decoherence effects, an RF cavity is used, with voltage $V = 100$ kV, and operating
frequency $f_{RF} = 5\cdot f_{rev}$, where $f_{rev} = 0.87$ MHz. Non-linear decoherence effects are suppressed
by using six sextupole families.

\begin{figure}[h!]
	\centering
	\includegraphics[width=\linewidth]{images/chapter2/6_3_lattice}
	\caption{QFS lattice design variant with spatially separated E- and B-fields.
          (Image taken from~\cite{Senichev:Lattices})\label{fig:QFS_6_3_lattice}}
\end{figure}

\subsubsection{QFS lattice design ``E+B''}\label{chpt2:lattice:QFS:EB}

The lattice design in Figure~\ref{fig:QFS_E+B_lattice} uses plain straight, static Wien filters.
This allows one to:
\begin{enumerate*}[\itshape a\upshape)]
	\item exclude non-linear electrostatic field components present in curved electrostatic fields, and 
	\item simplify the lattice from the engineering point of view.
\end{enumerate*}

The lattice is 149.21~m in length, the injection energy is 270 MeV. The linear spin decoherence effets suppressing
RF cavity has a longitudinal voltage $V = 100$ kV, and frequency $f_{RF} = 5\cdot f_{rev}$,
with $f_{rev} = 0.98$ MHz. Four sextupole families are used for the suppression of non-linear decoherence effects.
\begin{figure}[h!]
	\centering
	\includegraphics[width=\linewidth]{images/chapter2/E+B_lattice}
	\caption{Straight Wien filters QFS lattice variant.
          (Image taken from~\cite{Senichev:Lattices})\label{fig:QFS_E+B_lattice}}
\end{figure}

