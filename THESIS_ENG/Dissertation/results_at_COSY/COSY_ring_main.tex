
\newcommand{\Hm}{H$^-$}
\newcommand{\Dm}{D$^-$}

\begin{figure}[h]
	\centering
	\includegraphics[scale=.5]{images/chapter4/800px-COSY_Ring}
	\caption{Synchrotron COSY.\label{fig:COSY_Ring}}
\end{figure}

The COSY accelerator facility depicted in Figure~\ref{fig:COSY_Ring} consists of two sources of unpolarized \Hm/\Dm-ions and one source of polarized \Hm-ions, the injector cyclotron JULIC (J\"ulich Light Ion Cyclotron)~\cite{JULIC-Injector} capable of accelerating the \Hm-ions up to 300 MeV/c and \Dm-ions up to 600 MeV/c, and the cooler synchrotron ring COSY with a circumference 184 m accelerating protons and deuterons
up to 3.3 GeV/c.~\cite{COSY-Ring}

Injection into COSY is done via charge exchange of the negative ions over 20 ms with a linearly decreasing
closed orbit bump at the position of the stripper foil. The polarized source delivers 10 $\mu$A of polarized \Hm-ions.~\cite{COSY-Ring}

Two types of beam cooling are available: electron (energy range in the ``old'' and ``new'' electron coolers: 20--100 keV and 20--2,000 keV, respectively) and stochastic. 
%
The two electron coolers installed in the straight sections are capable of cooling the beam in the full possible energy range. Stochastic cooling works in the momentum range 1.5-3.7 GeV/c.

Beam polarization is continuously monitored  by an internal polarimeter EDDA; recently, an additional polarimeter making use of WASA forward detectors was set up, and a new polarimeter, based on LYSO-scintillators, is under development and will be installed in the COSY ring in 2019.
Proton polarization of 75\% can be achieved up to the highest momentum levels; deuteron vector and tensor polarizations reach up to 60\%.~\cite[\textbf{Historical background}]{YellowReport}