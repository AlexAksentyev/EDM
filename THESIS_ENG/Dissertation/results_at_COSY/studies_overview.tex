

\section{High precision spin tune measurement}
The collaboration has been able to achieve an unprecedented spin tune measurement sensitivity level 
of $10^{-10}$. 

The experiment consisted in the following. An initially vertically-polarized deuteron beam was injected into the ring. After the preparatory phase, during which the beam is cooled and bunched, the beam polarization was flipped into the horizontal plane by means of an RF solenoid inducing an imperfection resonance.~\cite[p.~7]{COSY:SpinTuneMapping}

After that, the beam was continuously extracted onto a carbon target, and the up-down 
cross section asymmetry was measured, which is proportional the beam's horizontal polarization component. Due to the specially designed data acquisition system~\cite{COSY:DAQ}, it was possible to precisely determine the number of turns the beam had done in the ring by the time an event was recorded on the detector.

The main problem with this measurement was that spin tune could not be estimated via 
a regular model fit to polarimetry data.
The spin precession frequency is approximately 120 kHz, while the detector sampling rate does not 
exceed 5 kHz, meaning that only one event could be detected per every 24 spin rotations about the vertical axis.
To solve the problem of data sparsity all measurements were mapped into one oscillation period.~\cite{Eversmann:SpinTuneMeasurement}

This allowed for the estimation of the spin tune at a precision level of $10^{-10}$ in a 100 second measurement cycle, which theoretically allows for the detection of the EDM at the sensitivity level $10^{-24}$\ecm.

\section{Beam Based Alignment}
\newcommand{\Nbpm}{N_{\mathrm{BPM}}}
Beam Based Alignment~\cite{Wagner:BBA2018} is a procedure to verify 
that the beam passes through the center of a quadrupole.
In order to so do, one varies the strength of the quadrupole and observes the changes this affects in the 
closed orbit. If the beam does not pass through the center of the quadrupole, the closed orbit will shift;
this shift can be described by
\[
\Delta x = \frac{\Delta k\cdot x(s_0)\ell}{B\rho}\cdot \frac{1}{1 - k\frac{\ell\beta(s_0)}{2B\rho\tan\pi\nu}}\cdot \frac{\sqrt{\beta(s)\beta(s_0)}}{2\sin\pi\nu}\cos\bkt{\phi(s) - \phi(s_0) - \pi\nu},
\]
where $\Delta x$ is the orbit change; $s$ is the coordinate of the beam position monitor; $s_0$ is the coordinate of the quadrupole; $\Delta k$ is the quadrupole strength change; $\ell$ is the quadrupole length; $\nu$ is the betatron tune; $\phi$ is the betatron phase; $x(s_0)$ the beam position with respect to the magnetic center of the quadrupole.

Since the orbit change $\Delta x(s)$ is a linear function of the offset of the beam with respect to the magnetic center of the quadrupole, one can determine the optimal position of the quadrupole by minimizing the function
\[
f = \frac{1}{\Nbpm}\sum_{i=1}^{\Nbpm} (x_i(+\Delta k) -x_i(-\Delta k)^2 \propto x^2(s_0).
\]

The first time, Beam Based Alignment was tested in the November-December 2017 beam time. The methodology requires that the strength of a single quadrupole is varied at a time, else the observed effect will be a superposition
of several closed orbit perturbations. Since quadrupoles at COSY are fed in groups of four, in order to vary the strength of a single quadrupole, additional coils on the poles of a quadrupole are used. In that case, 
the field of a quadrupole becomes a superposition of two quadrupole fields; however, this does not reflect on the methodology. 

During the measurement multiple different bumps were introduced into the closed orbit
at the quadrupole position. This leads to different magnitudes of the measured effect on the
closed orbit. Thereby multiple points could be scanned in horizontal and vertical direction
to find the optimal beam position inside the quadrupole.~\cite[p.~60]{Wagner:BBA2018}

The measurement was repeated in February 2019.

From a surveying procedure the quadrupole position is known to approximately 0.2 mm.~\cite[\textbf{Results and achievements at COSY}]{YellowReport}