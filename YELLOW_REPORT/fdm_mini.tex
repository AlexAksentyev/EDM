\documentclass{article}

\usepackage{amsmath}
\usepackage{xfrac}
\usepackage{paralist}
\usepackage{xparse}
\usepackage{url}
\usepackage{hyperref}

%% DEFINITIONS %%

\newcommand{\w}{\omega}
\newcommand{\nbar}{\bar n}
\DeclareDocumentCommand{\ddt}{m}{\frac{\mathrm{d} {#1}}{\mathrm{d} t}}
\DeclareDocumentCommand{\bkt}{sm}{\IfBooleanTF{#1}{\left[ #2 \right]}{\left(#2\right)}}
\newcommand{\D}{\Delta}
\DeclareDocumentCommand{\g}{s}{\gamma\IfBooleanT{#1}{_{eff}}}
\newcommand{\td}{\mathrm{d}}
\newcommand{\avg}[1]{\langle{#1}\rangle}
\newcommand{\wedm}{\w_{edm}}
\newcommand{\wimp}{\w_{\avg{E_v}}}
\newcommand{\wsw}{\w_{SW}}


\begin{document}
\title{Frequency Domain Method of search for the deuteron electric dipole moment}
\author{Y. Senichev, A. Aksentev, E. Valetov}
\date{\nodate}
\maketitle


We propose a method which aims at solving the geometric phase~\cite[p.~6]{BNL:Proton} and
machine imperfection~\cite[pp.~10,~11]{BNL:Deuteron2008} systematic errors,
encountered in any Frozen Spin Storage Ring EDM measurement method based on observation of a slow, gradual
change in the beam polarization vector.

Geometric phase can be handled by dispensing with operation in the spin resonance (i.e., 3D Frozen Spin) state,
in favor of the 2D FS state, generated by a Spin Wheel.~\cite[p.~1963]{Koop:SW}
In order to eliminate the machine imperfection systematic error,
we propose to utilize the imperfection fields themselves as a spin wheel.

Our method is intended for a combined storage ring (bend fields are magnetic).
Flipping of the spin wheel roll direction required by the SW methodology
is executed via reversing the guide field polarity.~\cite{Aksentev:IPAC19:GFF}~\cite{Aksentev:IPAC19:Decoh}
Control of its roll rate is achieved via observation of the polarization
precession frequency in the horizontal (closed orbit) plane.~\cite{Aksentev:IPAC19:GFF}

In more detail, the method is described in~\cite{Senichev:FDM}~\cite{Aksentev:FDM}.

\begin{thebibliography}{9}
\bibitem{BNL:Proton}
  V. Anastassopoulos et al., ``A Storage Ring Experiment to Detect a Proton Electric Dipole Moment.''
  Rev. Sci. Instrum., 87(11), 2016.
  \url{http://arxiv.org/abs/1502.04317}

\bibitem{BNL:Deuteron2008}
  D. Anastassopoulos et al., ``AGS Proposal: Search for a permanent electric dipole moment of
  the deuteron nucleus at the $10^{-29}$ e$\cdot$cm level,'' BNL, 2008.
  
\bibitem{Koop:SW}
   I. Koop,
   \textquotedblleft{Asymmetric Energy Colliding Ion Beams in the EDM Storage Ring}\textquotedblright,
   in \emph{Proc. 4th Int. Particle Accelerator Conf. (IPAC'13)}, Shanghai, China, May 2013, paper TUPWO040,
   pp. 1961--1963.
  \url{http://accelconf.web.cern.ch/accelconf/ipac2013/papers/tupwo040.pdf}

\bibitem{Aksentev:IPAC19:GFF}
  A. Aksentev, Y. Senichev, ``Simulation of the Guide Field Flipping Procedure for the Frequency Domain Method,'' 
  presented at the 10th International Particle Accelerator Conf. (IPAC'19), Melbourne, Australia,
  May 2019, paper MOPTS010.


\bibitem{Aksentev:IPAC19:Decoh}
  A. Aksentev, Y. Senichev, ``Spin decoherence in the Frequency Domain Method for the search of a particle EDM,''
  presented at the 10th International Particle Accelerator Conf. (IPAC'19), Melbourne, Australia,
  May. 2019, paper MOPTS012.

\bibitem{Senichev:FDM}
  Y.~Senichev, A.~Aksentev, A.~Ivanov and E.~Valetov,
  %``Frequency domain method of the search for the deuteron electric dipole moment in a storage ring
  %with imperfections,''
  arXiv:1711.06512 [physics.acc-ph].
  %%CITATION = ARXIV:1711.06512;%%
  \url{https://arxiv.org/abs/1711.06512}

\bibitem{Aksentev:FDM}
  A. Aksentev, Y. Senichev, E. Valetov, JEDI internal note \#4/2019.
  
\end{thebibliography}

\end{document}
