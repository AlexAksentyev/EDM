
\documentclass[a4paper]{jacow}

\usepackage{amsmath}
\usepackage{xfrac}
\usepackage{paralist}
\usepackage{xparse}

%% DEFINITIONS %%

\newcommand{\w}{\omega}
\newcommand{\nbar}{\bar n}
\DeclareDocumentCommand{\ddt}{m}{\frac{\mathrm{d} {#1}}{\mathrm{d} t}}
\DeclareDocumentCommand{\bkt}{sm}{\IfBooleanTF{#1}{\left[ #2 \right]}{\left(#2\right)}}
\newcommand{\D}{\Delta}
\DeclareDocumentCommand{\g}{s}{\gamma\IfBooleanT{#1}{_{eff}}}
\newcommand{\td}{\mathrm{d}}
\newcommand{\avg}[1]{\langle{#1}\rangle}
\newcommand{\wedm}{\w_{edm}}
\newcommand{\wimp}{\w_{\avg{E_v}}}
\newcommand{\wsw}{\w_{SW}}

% Senichev
\newcommand{\Wmdm}{\Omega_{\mathrm{mdm}}}
\newcommand{\Wedm}{\Omega_{\mathrm{edm}}}
\newcommand{\vWedm}{\vec{\Omega}_{\mathrm{edm}}}
\newcommand{\vWmdm}{\vec{\Omega}_{\mathrm{mdm}}}

\begin{document}


\title{Frequency domain method of search for the deuteron electric dipole moment}

\author{Yu. Senichev\textsuperscript{1}\thanks{y.senichev@inr.ru},
  A. Aksentev\textsuperscript{2,}\textsuperscript{3},
  E.Valetov\textsuperscript{4}, \\
  \textsuperscript{1} Institute for Nuclear Research of the Russian Academy of Sciences, Moscow, Russia\\
  \textsuperscript{2} Institut f\"ur Kernphysik, Forschungszentrum J\"ulich, J\"ulich, Germany\\
  \textsuperscript{3} National Research Nuclear University ``MEPhI,'' Moscow, Russia\\
  \textsuperscript{4} Department of Physics and Astronomy, Michigan State University, East Lansing, Michigan, USA}

\maketitle

\tableofcontents


\section{Motivation}

Storage ring-based methods of search for the electric dipole moments (EDMs) of fundamental particles
can be classified into two major categories, which we will call
\begin{inparaenum}[\itshape a\upshape)]
\item Space Domain, and % integral, cumulative methods: observe an integral characteristic --- accumulated angle
\item Frequency Domain % differential methods: observe an instantaneous characteristic --- angular velocity
\end{inparaenum}
methods.

In the Space Domain paradigm, one measures a \emph{change in the spatial orientation} of the beam polarization
vector \emph{caused by the EDM}.

The original storage ring, frozen spin-type method, proposed in~\cite{BNL:Deuteron2008}, is a canonical example of
a methodology in the space domain: an initially longitudinally-polarized beam is injected into the storage ring;
the vertical component of its polarization vector is observed. Under ideal conditions, any tilting of
the beam polarization vector from the horizontal plane is attributed to the action of the EDM.

Two technical difficulties are readily apparent with this approach:
\begin{enumerate}
\item it poses a challenging task for polarimetry~\cite{Mane:SpinWheel};
\item it puts very stringent constraints on the precision of the accelerator optical element alignment.
\end{enumerate}

The former is due to the requirement of detecting a change of about $5\cdot 10^{-6}$ to the
cross section asymmetry $\varepsilon_{LR}$ in order to get to the EDM sensitivity level
of $10^{-29}~e\cdot cm$.~\cite[p.~18]{BNL:Deuteron2008}

The latter is to minimize the magnitude of the vertical plane
MDM precession frequency:~\cite[p.~11]{BNL:Deuteron2008}
\begin{equation}\label{eq:BNL_syst_err}
\w_{syst} \approx \frac{\mu\avg{E_v}}{\beta c\gamma^2},
\end{equation}
induced by machine imperfection fields. According to estimates done by Y. Senichev, if it is to be fulfilled,
the geodetic installation precision of accelerator elements must reach $10^{-14}$ m. Today's technology
allows only for about $10^{-4}$ m.

At the practically-achievable level of element alignment uncertainty, $\w_{syst} \gg \wedm}$,
and changes in the orientation of the polarization vector are no longer EDM-driven.

Another crucial problem one faces in the space domain is geometric phase error.~\cite[p.~6]{BNL:Proton}
The problem here lies in the fact that, even if one can somehow make field imperfections (either due to
optical element misalignment or spurious electro-magnetic fields) zero
\emph{on average}, since spin rotations are non-commutative, the polarization rotation angle due to them
will not be zero.

By contrast, the Frequency Domain methodology is founded on measuring the EDM \emph{contribution} to the total
(MDM and EDM together) spin precession \emph{angular velocity}.

The polarization vector is made to roll about a nearly-constant, definite direction vector $\nbar$,
with an angular velocity that is high enough for its magnitude to be easily measureable at all times.
Apart from easier polarimetry, the definiteness of the angular velocity vector is a safeguard against geometric
phase error.

This ``Spin Wheel'' may be externally applied~\cite{Koop:SW}, or otherwise the machine imperfection fields
may be utilized for the same purpose (wheel roll rate determined by equation~\eqref{eq:BNL_syst_err}).
The latter is made possible by the fact that $\w_{syst}$ changes sign when the beam revolution direction
is reversed.~\cite[p.~11]{BNL:Deuteron2008}

\section{Universal SR EDM measurement problems}

By way of introduction to the proposed measurement methodology, let us briefly summarize some measurement problems
encountered by any EDM experiment performed in a storage ring; they can be grouped into two big categories:
\begin{itemize}
\item Problems solved by a Spin Wheel:
  \begin{itemize}
  \item spurious electro-magnetic fields;
  \item betatron motion.
  \end{itemize}
\item Problems having specific solutions:
  \begin{itemize}
  \item spin decoherence;
  \item machine imperfections.
  \end{itemize}
\end{itemize}

\subsection{Spin Wheel-solvable problems}
Problems from the first category are ones introducing geometric phase error. Indeed, both the spurious 
and the focusing fields, when acting on a betatron-oscillating particle, perturb the direction and
magnitude of its spin precession angular velocity vector. The effect is a spin kick in the direction defined
by the perturbation.

Assume that the EDM provides a spin kick about the radial ($\hat x$) axis. The magnitude of the angular
velocity vector has a general form
\[
\w = \sqrt{\w_x^2 + \w_y^2 + \w_z^2},
\]
where $\w_y$ is minimized by fulfilling the frozen spin condition; $\w_z$ (the constant part of which is
due to machine imperfections) can be minimized via the installation
of a longitudinal solenoid on the optic axis.\footnote{1 m long, magnetic field approximately $10^{-6}$ T.} In the
space domain, one also tries to minimize the $\wimp$ contribution to $\w_x = \wedm + \wimp$. Consequently,
spin kicks must be minimized to (significantly) less than $\wedm$, to reduce geometric phase to
less than the accumulated EDM phase.

The benefit of having a Spin Wheel aligned with the EDM angular velocity is that orthogonal MDM contributions
to the angular velocity vector add up in squares, and hence their effect is greatly diminished:
\begin{align*}
  \w &= \sqrt{(\wedm + \wsw)^2 + \w_y^2 + \w_z^2} \\
  &\approx(\wedm + \wsw)\cdot \bkt*{1 + \frac{\w_y^2 + \w_z^2}{\wsw^2}}^{\sfrac12} \\
  &\approx (\wedm + \wsw)\cdot \bkt{1 + \frac{\w_y^2 + \w_z^2}{2\wsw^2}} \\
  &\approx \wsw + \wedm + \underbrace{\frac12\frac{\w_y^2 + \w_z^2}{\wsw}}_{\epsilon}.
\end{align*}

Since our goal is to observe the EDM-related value shift in $\w$, we need to minimize random variable
$\epsilon$:
\[
\frac12\frac{\w_y^2 + \w_z^2}{\wsw} < \wedm.
\]

Let's make some preliminary estimates. Suppose $\wsw\approx 50$ rad/sec (the reason for choosing this
value will be explained shortly), $\wedm\approx10^{-9}$ rad/sec (corresponding to the EDM value
$10^{-29}~e\cdot$ cm). Then, $\w_y^2 + \w_z^2$ must be reduced to less than $10^{-7}$ rad/sec, or equivalently,
either angular velocity to less than $3\cdot 10^{-4}$ rad/sec. This is several orders of magnitude greater than
the expected standard error on the angular velocity estimate,~\cite{Aksentev:Stats} and hence
should not be a problem to achieve.

One case left to be considered is MDM spin kicks about the $\hat x$ axis. These are not attenuated, and cause the
most trouble. They come in three varieties:
\begin{inparaenum}[\itshape a\upshape)]
\item permanent, not caused by optical element misalignments;
\item semi-permanent, caused by element tilts about the optic axis;
\item spurious.
\end{inparaenum}

Semi-permanent spin kicks (be they caused by magnetic or electric fields) change sign when
the beam revolution direction is reversed from clockwise (CW) to counter-clockwise(CCW).
Spurious kicks can be dealt with by statistical averaging.
Permanent insensitive to either the guide field or the beam circulation direction cannot be controlled.
On the bright side, their sources should not be present under normal circumstances.

\subsection{Expected machine imperfection SW roll rate}
In the estimates above, we used a roll rate $\wsw\approx 50$ rad/sec for the spin wheel. This value comes
from the following considerations.

CONSIDERATIONS.....

\subsection{Spin decoherence}
Spin coherence is a measure or quality of preservation of polarization in an initially fully-polarized
beam.~\cite{Eremey:Thesis} Spin decoherence refers to the depolarization caused by the difference in the
beam particles' spin precession frequencies. 

The difference in spin tunes is due to the difference of the particles' orbit lengths, and hence their
equilibrium energy levels, on which spin tune depends. One way spin decoherence can be suppressed is by
utilization of sextupole fields. We consider how this can be accomplished in~\ref{Aksentev:IPAC19:Decoh}.

\subsection{Machine imperfections}

As we have seen, the problem with machine imperfections is twofold:
\begin{inparaenum}[\itshape a\upshape)]
\item they are prcatically impossible to remove at the present level of technology, but what's even worse 
\item their removal leaves one in the space domain, and opens the measurement up to geometric phase error.
\end{inparaenum}

Fortunately for us, the imperfection spin kicks they induce change sign when the beam circulation direction
is reversed. Their magnitude is also sufficient for their use as a Koop Wheel. The one remaining difficulty
is the accuracy of the Koop wheel roll direction flipping. Hopefully, we can make a persuasive
enough argument as to how this can accomplished.

\section{Measurement methodology main features}

The method we propose is characterized by two main features:
\begin{enumerate}
\item It is a frequency domain method;
\item The fields induced by machine imperfections, instead of being suppressed,
  are used as a Koop Wheel;
  \begin{itemize}
  \item The Koop Wheel roll direction is reversed by flipping the direction of the guide field;
  \item its roll rate is controlled through observation of the horizontal plane
    polarization precession frequency.
  \end{itemize}
\end{enumerate}

The advantages of the frequency domain, such as
\begin{inparaenum}[\itshape a\upshape)]
\item ease of polarimetry, and
\item immunity to geometric phase error,
\end{inparaenum}
have been highlighted in prevous sections. Now we will turn to the description of how machine imperfection fields
can be used as a Koop Wheel.

\section{EDM estimator statistic}
Since the angular velocity measured in the frequency domain methodology includes contributions due to both the
magnetic and electric dipole moments, the EDM estimator statistic requires two cycles to compose:
one in which the Koop Wheel rolls forward, the other backward.

The change in the Koop Wheel roll direction is affected by flipping the direction of the guide field.
When this is done:
$\vec B \mapsto -\vec B$, the beam circulation direction changes from clockwise (CW) to counter-clockwise (CCW): 
$\vec\beta \mapsto -\vec\beta$, while the electrostatic field remains constant: $\vec E \mapsto \vec E$.
According to the T-BMT equation, spin precession frequency components change like:
\begin{subequations}
  \begin{align}
    \w_x^{CW} &= \w_x^{MDM, CW}   + \w_x^{EDM}, \notag\\
    \w_x^{CCW} &= \w_x^{MDM, CCW} + \w_x^{EDM}, \notag\\
    \w_x^{MDM, CW} &= -\w_x^{MDM, CCW}, \label{eq:CW_CCW_MDM}\\
    \intertext{and the EDM estimator}
    \hat\w_x^{EDM} &:= \frac12\bkt{\w_x^{CW} + \w_x^{CCW}} \label{eq:FDM_estimator} \\
                  &=  \w_x^{EDM} +
          \underbrace{\frac12\bkt{\w_x^{MDM, CW} + \w_x^{MDM, CCW}}}_{\varepsilon \to 0}.
  \end{align}
\end{subequations}

To keep the systematic error term $\varepsilon$ below required precision, i.e. ensure
that equation~\eqref{eq:CW_CCW_MDM} holds with sufficient accuracy, we devised a guide field flipping procedure
based on observation of the horizontal plane spin precession frequency.

To explain how it works, we need to introduce the concept of the effective Lorentz factor.

\section{Effective Lorentz factor}
Spin dynamics is described by the concepts of \emph{spin tune} $\nu_s$ and \emph{invariant spin axis} $\nbar$.
Spin tune depends on the the particle's  equilibrium-level energy, expressed by the Lorentz factor:
\begin{equation}\label{eq:spin_tune_vs_gamma}
  \begin{cases}
    \nu_s^B &= \gamma G, \\
    \nu_s^E &= \beta^2\gamma\bkt{\frac{1}{\gamma^2-1} - G} \\
            &= \frac{G+1}{\gamma} - G\gamma.
  \end{cases}
\end{equation}

Unfortunately, not all beam particles share the same Lorentz factor. A particle involved in betatron
motion will have a longer orbit, and as a direct consequence of the phase stability principle,
in an accelerating structure utilizing an RF cavity, its equilibrium energy level 
must increase. Otherwise it cannot remain the bunch. In this section we analyze how the particle Lorentz factor
should be modified when betatron motion, as well as non-linearities in the momentum compaction factor are
accounted for.

The longitudinal dynamics of a particle on the reference orbit of a storage ring is described
by the system of equations:
\begin{equation}
  \begin{cases}
    \ddt{}\D\varphi &= -\w_{RF}\eta\delta, \\
    \ddt{}\delta &= \frac{q V_{RF}\w_{RF}}{2\pi h\beta^2E}\bkt{\sin\varphi - \sin\varphi_0}.
  \end{cases}
\end{equation}
In the equations above, $\D\varphi = \varphi - \varphi_0$ and
$\delta = \bkt{p-p_0}/{p_0}$ are the deviations of the particle's phase and
normalized momentum from those of the reference particle; all other symbols have their usual meanings.
%% $V_{RF}$, $\w_{RF}$ are, respectively,
%% the RF voltage and frequency; $\eta = \alpha_0 - \gamma^{-2}$ is the slip-factor,
%% where $\alpha_0$ is the momentum compaction factor defined by $\sfrac{\Delta L}{L} = \alpha_0\delta$,
%% $L$ being the orbit length; $h$ is the harmonic number; $E$ the total energy of the particle.

The solutions of this system form a family of ellipses in the $(\varphi, \delta)$-plane, all centered at the
point $(\varphi_0,\delta_0)$. However, if one considers a particle involved in betatron oscillations, and
uses a higher-order Taylor expansion of the momentum compaction factor
$\alpha = \alpha_0 + \alpha_1\delta$, the first equation of the system
transforms into:~\cite[p.~2579]{Senichev:IPAC13}
\begin{align*}
  \ddt{\D\varphi} = -\w_{RF} \Bigg[\bkt{\frac{\Delta L}{L}}_\beta &+ \bkt{\alpha_0 + \gamma^{-2}}\delta \right.\\
    &+ \left.\bkt{\alpha_1 - \alpha_0\gamma^{-2} + \gamma^{-4}}\delta^2\Bigg],
\end{align*}
where $\bkt{\frac{\Delta L}{L}}_\beta = \frac{\pi}{2L}\bkt*{\varepsilon_xQ_x + \varepsilon_yQ_y}$, is
the betatron motion-related orbit lengthening; $\varepsilon_x$ and $\varepsilon_y$ are
the horizontal and vertical beam emittances, and $Q_x$, $Q_y$ are the horizontal and vertical tunes.

The solutions of the transformed system are no longer centered at the same single point. Orbit lengthening
and momentum deviation cause an equilibrium-level momentum shift~\cite[p.~2581]{Senichev:IPAC13}
\begin{equation}\label{eq:EquLevMom_shift}
\Delta\delta_{eq} = \frac{\gamma_0^2}{\gamma_0^2\alpha_0 - 1}\bkt*{\frac{\delta_m^2}{2}\bkt{\alpha_1 - \alpha_0\gamma^{-2} + \gamma_0^{-4}} + \bkt{\frac{\Delta L}{L}}_\beta},
\end{equation}
where $\delta_m$ is the amplitude of synchrotron oscillations.

We call the equilibrium energy level associated with the momentum shift~\eqref{eq:EquLevMom_shift},
the \emph{effective Lorentz factor}:
\begin{equation}\label{eq:EffectiveGamma}
\g*= \gamma_0 + \beta_0^2\gamma_0\cdot\Delta\delta_{eq},
\end{equation}
where $\gamma_0$, $\beta_0$ are the Lorentz factor and relative velocity factor of the reference particle.

Observe, that the effective Lorentz factor enables us to account for variation in the value of spin tune
due to variation in the particle orbit length. It is crucial in the analysis of
spin decoherence~\cite{Aksentev:IPAC19:Decoh} and its suppression by means of sextupole fields.

It plays a big role, as well, in the successfull reproduction of the MDM component to the total spin precession
angular velocity.

\section{Guide field flipping}
\newcommand{\Traj}{\mathcal T}
\DeclareDocumentCommand{\Stab}{s}{\mathcal{S}\IfBooleanT{#1}{\vert_{\w_y=0}}}
\newcommand{\Fail}{\mathcal F}
\renewcommand{\D}{\mathcal D}

Two aspects of the problem need to be paid attention to:
\begin{enumerate}
\item What needs to be kept constant from one measurement cycle to the next;
\item How it can be observed.
\end{enumerate}

The goal of flipping the direction of the guide field is to accurately reproduce the radial component
of the MDM spin precession frequency induced by machine imperfection fields. This point should not be overlooked:
a mere reproduction of the \emph{magnetic field strength} would not suffice, since the injection point of the beam's centroid,
and hence its orbit length --- and, via equations~\eqref{eq:EffectiveGamma} and~\eqref{eq:spin_tune_vs_gamma}, spin tune, --- is subject to variation. (Apart from that, the accelerating structure might not be symmetrical, in terms of spin dynamics, with regard to reversal of the beam circulation direction.)

What needs to be reproduced, therefore, is not the field strength, but the effective Lorentz factor of the centroid.

Regarding the second question, we mentioned earlier that the Koop Wheel roll rate
is controlled through measurement of the horizontal plane spin precession frequency. 
This plane was chosen because the EDM angular velocity vector points
(mainly) in the radial direction; its vertical component is due to machine imperfection fields, and is small compared to
the measured EDM effect. Therefore, in first approximation, when we manipulate the vertical component of the 
combined spin precession angular velocity, we manipulate the vertical component of the MDM angular velocity vector.

Moving on to the effective Lorentz factor calibration procedure.
Let $\Traj$ denote the set of all trajectories that a particle might follow in the accelerator.
$\Traj = \Stab \bigcup \Fail$, where $\Stab$ is the set of all stable trajectories, $\Fail$ are all trajectories
such that if a particle gets on one, it will be lost from the bunch.

Calibration is done in two phases:
\begin{enumerate}
\item In the first phase, the guide field value is set so that the beam particles are injected onto trajectories
  $t\in\Stab$.
\item In the second phase, it is fine-tuned further, so as to fulfill the FS condition in the horizontal plane.
  By doing this, we physically move the beam trajectories into the subset $\Stab*\subset\Stab$ of trajectories 
  for which $\w_y = 0$.
\end{enumerate}

Spin tune (and hence precession frequency) is an injective function of the
effective Lorentz-factor $\g*$, which means
$\w_y(\g*^1) = \w_y(\g*^2) \rightarrow \g*^1 = \g*^2$. The trajectory space $\Traj$ is partitioned into equivalence
classes according to the value of $\g*$: trajectories characterized by the same $\g*$ are equivalent
in terms of their spin dynamics (possess the same spin tune and invariant spin axis direction),
and hence belong to the same equivalence class.
Since $\w_y(\g*)$ is injective, there exists a unique $\g*^0$ at which $\w_y(\g*^0)=0$:
\[
[\w_y=0] = [\g*^0] \equiv \Stab*.
\]

If the lattice didn't use sextupole fields for the suppression of decoherence,
$\Stab*$ would be a singleton set. We have shown in~\cite{Aksentev:IPAC19:Decoh} that if sextupoles are
utilized, then $\exists\D\subset\Stab$ such that $\forall t_1,t_2\in\D$:
$\nu_s(t_1) = \nu_s(t_2)$, $\nbar(t_1) = \nbar(t_2)$. By adjusting the guide field strength we equate
$\D=\Stab*$, and hence $\Stab*$ contains multiple trajectories.~\footnote{Strictly speaking,
  even if sextupoles are used there remains some negligible dependence of spin tune
  on the particle orbit length (linear decoherence effects, cf.~\cite{Aksentev:IPAC19:Decoh}).
  Because of that, the equalities for $\nu_s$ and $\nbar$ are approximate, and the set $\Stab*$
  should be viewed as fuzzy:
  we will consider trajectories for which $|\w_y|<\delta$ for some small $\delta$ as belonging to $[\w_y=0]$.}

Therefore, once we ensured that the beam polarization does not precess in the horizontal plane,
all of the beam particles have $\g*^0$, equal for the CW and CCW beams.

Guide field flipping procedure simulation results can be found in~\ref{Aksentev:IPAC19:GFF}.


\section{Statistical precision}
 
First, we see in~\eqref{eq:Omega CW_CCW} that the accuracy of the frequency measurements
of $\Omega_{r}^{\mathrm{CW}}$ and $\Omega_{r}^{\mathrm{CCW}}$   determines the precision of the EDM measurement.
In~\cite{Eversmann}, it is shown that the relative accuracy of the polarization precession frequency
measurement, $10^{-10}$ to $10^{-11}$,   is achievable even when the frequency of polarization measurements (a detector rate) is much less than the polarization precession frequency. In our case, we have an inverse relationship between the polarimeter rate and the measured spin frequency, which extends the range of frequencies where statistical estimates are legitimate. As shown in ~\cite{Aksentyev}, for an absolute statistical error of measuring a frequency of the spin oscillation, we can use $\sigma_{\Omega}=\delta\epsilon_{A}\sqrt{24/N}/T$, where $N$  is the total number of recorded events, $\delta\epsilon_{A}$   is the relative error in measuring the asymmetry, and $T\approx 1000$  sec is the measurement duration. If we assume a beam of $10^{11}$  particles per fill and a polarimeter efficiency of one percent, this leads to an absolute error of frequency measurement of  $\sigma_{\Omega}=2\cdot10^{-7}$ rad/sec. With a nominal accelerator beam time of 6,000 hours per year, we can reach $\sigma_{\Omega}=2\cdot10^{-9}$ rad/sec during one year. If we take into account that formula~\eqref{eq:Wedm} with the EDM $d_d\approx 10^{-30} e\cdot cm$  gives a value of the spin precession frequency of $\Omega_{\mathrm{edm}}\approx 10^{-8}$, we can state that the accuracy for the frequency of $\sigma_{\Omega}=1.4\cdot10^{-9}$   is satisfactory and sufficient for reaching a sensitivity of $d_d\approx 10^{-30}e\cdot cm$ (where $\eta\approx2\cdot10^{-15}$).

% Second, the main idea behind using CW and CCW procedures is that the contribution of the MDM spin rotation is the same for both CW and CCW directions. In an ideal scenario, the difference $\Omega_{r,\mathrm{mdm}}^{\mathrm{CCW}}-\Omega_{r,\mathrm{mdm}}^{\mathrm{CW}}$   is zero. However, this is not exactly the case. In reality, we do not know how accurately the field is recovered after a change of polarity, that is to say whether the energy of the beam is the same or not. Furthermore, the CW and CCW beam trajectories may  have different orbit lengths, which in turn contribute to the MDM spin precession frequency. We must therefore reformulate the global problem regarding how to restore the conditions for the equal contribution of the two MDM spin rotations after a change in the polarity of magnetic field (no change for electrical field) in the plane where we will measure the EDM. We expect to achieve a difference $\Omega_{r,\mathrm{mdm}}^{\mathrm{CCW}}-\Omega_{r,\mathrm{mdm}}^{\mathrm{CW}}$ that is smaller than the expected EDM precession frequency   $\Omega_{\mathrm{edm}}$ (see eq. 7). In this regard, we will undertake two procedures. The first follows from the study of the suppression of the spin decoherence~\cite{IPAC2013,decoherence}, where we reached a very important conclusion, namely that two arbitrary particles will have the same spin tune, independently of initial condition if their  orbits in 3D space have the same length. We refer to this as the conditions of zero decoherence of the spin precession.


%  Equation~\eqref{eq:Omega CW_CCW} quoted from~\cite{IPAC2013} shows this dependence:
% \begin{equation}\label{eq:delta p}
% \Delta\delta_{eq}=\frac{\gamma_{s}^{2}}{\gamma_{s}^{2}\alpha_0-1}\left[\frac{\delta_{m}^{2}}{2}\left(\alpha_{1}-\frac{\alpha_{0}}{\gamma_{s}}^{2}+\frac{1}{\gamma_{s}^{4}}\right)+\left(\frac{\Delta L}{L}\right)_{\beta}\right]	
% 	\end{equation}
% where $\Delta\delta_{eq}$  is the relative deviation of the equilibrium level (average value) of the momentum due to the orbit increasing in length in the transverse plane $\left(\frac{\Delta L}{L}\right)_{\beta}$   for a synchrotron oscillation with amplitude $\delta_{m}$ of relative momentum. Values $\alpha_{0}$ and $\alpha_{1}$  are the zero and first order momentum compaction factors, while $\gamma_{s}$  is the Lorentz factor of the synchronous particle. 

% The new equilibrium factor Lorentz,
% \begin{equation}\label{eq:effec gamma}
% \gamma_{eff}=\gamma_{s}+\beta_s^{2} \gamma_s\cdot\Delta\delta_{eq}	
% \end{equation}
% will hereinafter be referred to as the effective Lorentz factor. This parameter is named as such because $\gamma_{eff}$ includes three spatial coordinates and completely determines the frequency of spin precession in all three planes. The orbit length of each particle is adjusted by the sextupoles, leading to a dependence of the action of the sextupole field on the amplitude of transverse oscillations and the energy deviation from the reference particle. We can now apply this important conclusion to the beams moving in opposite CW and CCW directions, namely that the beams are identical in terms of the spin behavior if they have the same effective Lorentz factor averaged over all particles in beam. This means that the problem of finding the multiparameter dependence of spin precession on fields and 3D trajectories is reduced to the search for a dependence on the effective gamma. This ensures it is no longer necessary to obtain a coincidence of trajectories, but instead only requires the condition of equality $\gamma_{eff}$   for the CW and CCW beams. This approach saves the whole idea of searching for an EDM in a storage ring.


% Third, if we assume that there are two rings with a direct (CW) and reverse sequence of elements (CCW) with a changed polarity of the magnetic field, the similarity of these rings under the beam stability condition~\eqref{eq:Fav} is only that the position of all elements on the ring and, consequently, the relation between the values of the vertical and radial components of the field remains unchanged
% \begin{equation}\label{eq:element position}
% B_r/B_v=const\ \text {and}\ E_v/E_r=const
% \end{equation}
% Here, we should take into account the two facts mentioned above: first, we will not change the polarity of the electric field, leaving it unchanged during the transition from CW to CCW, and second, when the energy rises, the condition of equilibrium for the particles will be maintained by changing the magnetic field only. In our case this is a change in the magnetic field from 0.3 to 0.45 Tesla. Therefore, we will in the future only focus our attention on the magnetic field. Irrespective of this circumstance, it is unlikely that the reverse trajectory will coincide with the direct trajectory, which can be a reason for having a different orbit length and, hence, different Lorentz factor values   that determine the spin precession frequency in all planes. Therefore, before changing the polarity, we must calibrate the gamma $\gamma_{eff}$   close to the value $\gamma\approx\gamma_s$   using the precession frequency measurements of the spin in the horizontal plane where there is no EDM signal before restoring the same $\gamma_{eff}$   in the ring with the reverse sequence of elements.

% For such a calibration, we need to reduce the spin oscillation in the vertical plane to a low value by introducing the Wien filter 1 m long with orthogonal horizontal magnetic $\vec{B}$  and vertical electric
% $\vec{E}$ fields in the order of 0.1 mT and 100 V/cm respectively. The Wien filter installed in a straight section provides zero Lorentz force on axis and orientates spin in a horizontal plane. 
%  The value of this field does not affect the calibration of the effective Lorentz factor. Here, we are aiming to observe how to slow down the spin rotation in the vertical plane which it mean that precise knowledge of the fields is not needed. The sole purpose of introducing the Wien filter is to ensure that the relative contribution of the vertical frequency into the horizontal frequency is less than the calibration accuracy required, namely $ 10^{-9}$. Since they add up as squares of frequencies, this can be easily achieved. The transverse spin rotator is switched on only for the time of calibration of the $\gamma_{eff}$ in the CW ring and for the time of its recovery in the CCW ring. We are able to calibrate the frequency, $\gamma_{eff}$, with the above-mentioned absolute value of errors for one beam fill of $\sigma_{\Omega}\approx10^{-7}$ rad/sec and $\sigma_{\Omega}\approx10^{-9}$ rad/sec with one year of running. Taking into account the constant relation between the vertical and the radial components of field~\eqref{eq:element position}, this means that in the case of CCW we have a ring identical to the CW ring in terms of spin behavior, and we can obtain a zero value of $\Omega_{r,\mathrm{mdm}}^{\mathrm{CCW}}-\Omega_{r,\mathrm{mdm}}^{\mathrm{CW}}$    with an accuracy of $\approx 10^{-9}$.

% Finally, we will consider the fourth important aspect in the proposed procedure for measuring the EDM. This problem concerns the fact that the spin oscillating around an arbitrary axis always has a mixing of spin oscillation relative to other axes. The solution of equation~\eqref{eq:Wedm} under the initial conditions for horizontal, vertical ($S_x=0, S_y=0$), and longitudinal  $S_z=1$ components  can be formulated as shown here:
% \begin{align}
% S_x=\frac{\Omega_x\Omega_z}{\Omega^{2}}\left(1-\cos\Omega t\right)-\frac{\Omega_y}{\Omega}\sin\Omega t, \notag\\
% S_y=\frac{\Omega_y\Omega_z}{\Omega^{2}}\left(1-\cos\Omega t\right)+\frac{\Omega_x}{\Omega}\sin\Omega t, \notag\\
% S_z=\frac{\Omega_z^{2}}{\Omega^{2}}\left(1-\cos\Omega t\right)+\cos\Omega t,
% \end{align}
% where $\Omega_x=\Omega_{B_r}+\Omega_{edm}$   and $\Omega_z=\Omega_{B_z}$  arise due to MDM rotation in the imperfect ring and the EDM. $\Omega_{y}=\Omega_{B_{v},E_{r}}$  is the MDM spin rotation relative to the momentum in the leading magnetic and electric fields, and $\Omega=\sqrt{\Omega_x^{2}+\Omega_y^{2}+\Omega_z^{2}}$   is the modulus of the three-dimensional frequency. As mentioned above, we will measure the precession frequency of the spin in a vertical plane in order to study the behavior of the oscillating part of $\widetilde{S_y}$, the solution to which is as follows:
% \begin{align}
% \widetilde{S_y}=\sqrt{\left(\frac{\Omega_y\Omega_z}{\Omega^{2}}\right)^{2}+\left(\frac{\Omega_x}{\Omega}\right)^{2}}\sin\left(\Omega t+\phi\right), \notag\\
% \phi=\arctan\left(\frac{\Omega_y\Omega_z}{\Omega_x\Omega}\right),
% \end{align}
% Since the amplitude and the phase of the signal do not affect the measurement, we are only interested in the frequency:
% \begin{equation}\label{eq:Omega}
% \Omega=\sqrt{\left(\Omega_{edm}+\Omega_{B_r}\right)^{2}+\Omega_{B_v,E_r}^{2}+\Omega_{B_z}^{2}}	
% \end{equation}
% Assuming that, in accordance with the ``frozen'' spin concept, we maintain the spin along the momentum $\Omega_{B_v,E_r}<<\Omega_{B_r}$  and $\Omega_{B_z}<<\Omega_{B_r}$, the latter expression is realized by installing a solenoid with a longitudinal axis one meter long on a straight section with a magnetic field of about $\approx 10^{-6}$  Tesla, which can be formulated as follows:
% \begin{equation}\label{eq:Omega2}
% \Omega=\left(\Omega_{edm}+\Omega_{B_r}\right)\cdot\left[1+\frac{\Omega_{B_v,E_r}^{2}+\Omega_{B_z}^{2}}{2\left(\Omega_{edm}+\Omega_{B_r}\right)^{2}}\right]	
% \end{equation}
% According to this equation, the restriction occurs at the values of $\Omega_{B_v,E_r}$   and $\Omega_{B_z}$, which should have less of an effect on the total frequency $\Omega$   than the EDM:
% \begin{equation}\label{eq:Omega3}
% \frac{\Omega_{B_v,E_r}^{2}+\Omega_{B_z}^{2}}{2\Omega_{B_r}}<\Omega_{edm}\end{equation}
% If we evaluate these requirements numerically, we can assess how feasible it is to implement them technically. For instance, if $\Omega_{B_r}\approx$ 100 rad/sec and $\Omega_{edm}\approx10^{-8}$  rad/sec, then $\Omega_{B_v,E_r}^{2}+\Omega_{B_z}^{2}<10^{-6}$   or both must be $\Omega_{B_v,E_r}, \Omega_{B_z}\sim10^{-3}$ rad/sec. This means that at the spin coherence time $t_{SCT}\sim$ 1000 sec, the spin rotation should not exceed $\Omega_{B_v,E_r}\cdot t_{SCT}\sim$  1 rad and $\Omega_{B_z}\cdot t_{SCT}\sim$  ~1 rad, which is easily achievable both for $\Omega_{B_v,E_r}$   due to the calibration of energy and for $\Omega_{B_z}$   due to the introduction of a solenoid with a longitudinal magnetic field ${B_z}$. As in the case of a transverse spin rotator, the longitudinal field in the solenoid does not need to be known exactly, since it is only needed to satisfy equation~\eqref{eq:Omega3}, which is an approximation. We can therefore conclude that the imperfections of ring elements, which previously played a limiting role in the measurement of EDM due to the effect of so called geometrical phase \cite{AGS_proposal}, now provide a pure precession of the spin in the vertical plane, where we will measure the EDM.

% The displacement of the magnetic quadrupoles can also lead to the appearance of a dipole field component that induces a fake signal, which requires an identity for CW and CCW. However, at least two solutions exist here. The first is to use electrostatic quadrupoles. The second is to use optics that does not require switching the polarity of the magnetic field in magnetic quadrupoles.

% In this Letter, we described the frequency domain method of the search for the deuteron electric dipole moment in a storage ring with imperfections. The method differs from the one~\cite{Farley,AGS_proposal} that was previously proposed in that we use the measurement of the frequency of the total EDM and MDM signal, as opposed to the value of the vertical component of spin. An important part of this method is the introduction and measurement of the effective Lorentz factor. It determines the precession of the spin in 3D space, instead of controlling the three-dimensional orbital motion. This method allows to reduce the influence of systematic errors to the level at which the lower limit of detection of the assumed EDM can be as low as $\sim 10^{-29}\div 10^{-30}e\cdot cm$. 

% A significant part of the work was done when the authors worked in FZJ, and they would like to thank the JEDI collaboration and especially H. Stroeher E. J. Stephenson, V. Hejny, J. Pretz, F. Rathmann, A. Kacharava, K. Nikolaev,  A.Saleev and R. M. Talman for their scientific critical remarks, helping to make this work successful.

\begin{thebibliography}{9}
%% \bibitem{Canetti}
%%   Canetti L., Drewes M., Shaposhnikov M. (2012). ``Matter and Antimatter in the Universe,'' New Journal of Physics, 14, 095012.
%% \bibitem{Aguilar}
%%   Aguilar M., Alberti G., Alpat B. et al., (2013) ``First Result from the Alpha Magnetic Spectrometer on the International Space Station: Precision Measurement of the Positron Fraction in Primary Cosmic Rays of 0.5 - 350 GeV,'' Phys. Rev. Lett., 110, 141102.
%% \bibitem{Sakharov}
%%   Sakharov A. (1967). ``Violation of CP Invariance, C Asymmetry, and Baryon Asymmetry of the Universe,'' Letters to Jounal of Experimental and Theoretical Physics, 5, 24- 26.
%% \bibitem{Farley}
  %%   F. J. M. Farley, K. Jungmann, J. P. Miller, W.M. Morse, Y. F. Orlov, B. L. Roberts, Y. K. Semertzidis, A. Silenko  and E. J. Stephenson, ``New Method of Measuring Electric Dipole Moments in Storage Rings,'' Phys. Rev. Lett. 93, 052001 (2004).
  
\bibitem{BNL:Deuteron2008}
  D. Anastassopoulos et al., ``AGS Proposal: Search for a permanent electric dipole moment of
  the deuteron nucleus at the $10^{-29}$ e$\cdot$cm level,'' BNL, 2008.

\bibitem{Mane:SpinWheel}
  S. Mane, ``A distillation of Koop's idea of the Spin Wheel,'' arXiv:1509.01167 [physics]
  \url{http://arxiv.org/abs/1509.01167}.

\bibitem{BNL:Proton}
  V. Anastassopoulos et al., ``A Storage Ring Experiment to Detect a Proton Electric Dipole Moment.''
  Rev. Sci. Instrum., 87(11), 2016.
  \url{http://arxiv.org/abs/1502.04317}

\bibitem{Koop:SW}
  I. Koop. ``Asymmetric energy colliding ion beams in the EDM storage ring,'' Proc. of IPAC13 (2013).
  \url{http://accelconf.web.cern.ch/accelconf/ipac2013/papers/tupwo040.pdf}.

\bibitem{Senichev:IPAC13}
  Y. Senichev et al., ``Spin tune decoherence effects in Electro- and Magnetostatic Structures.''
  Proceedings of IPAC 2013, Shanghai, China, pp. 2579-2581

\bibitem{Senichev:FDM}
  Y. Senichev, A. Aksentev, A. Ivanov, E. Valetov, ``Frequency domain method of the search for
  the deuteron electric dipole moment in a storage ring with imperfections,'' arxiv:1711.06512 [physics.acc-ph]
  \url{https://arxiv.org/abs/1711.06512}.

\bibitem{Aksentev:Stats}
  A. Aksentev, Y. Senichev 2017 J. Phys.: Conf. Ser. 941 012083

\bibitem{Aksentev:IPAC19:SMP}
  A. Aksentev, Y. Senichev, ``Spin Motion Perturbation Effect on the EDM Statistic
  in the Frequency Domain Method,'' presented at the 10th International Particle Accelerator Conf. (IPAC'19),
  Melbourne, Australia, May. 2019, paper 2743.

\bibitem{Aksentev:IPAC19:Decoh}
  A. Aksentev, Y. Senichev, ``Spin decoherence in the Frequency Domain Method for the search of a particle EDM,''
  presented at the 10th International Particle Accelerator Conf. (IPAC'19), Melbourne, Australia,
  May. 2019, paper 2738.

\bibitem{Aksentev:IPAC19:GFF}
  A. Aksentev, Y. Senichev, ``Simulation of the Guide Field Flipping Procedure for the Frequency Domain Method,'' 
  presented at the 10th International Particle Accelerator Conf. (IPAC'19), Melbourne, Australia,
  May 2019, paper 2750.

\bibitem{Eremey:Thesis}
  E. Valetov, ``Field modeling, symplectic tracking, and spin decoherence for the EDM and muon g-2 lattices.''
  PhD tehsis, Michigan State University, Michigan, USA.
  \url{http://collaborations.fz-juelich.de/ikp/jedi/public_files/theses/valetovphd.pdf}
  
%% \bibitem{ICAP2015}
%%   Yu.Senichev et al., ``Investigation of Lattice for Deuteron EDM Ring,'' Proceedings of ICAP2015, Shanghai, China

%% \bibitem{Berz}
%%   M. Berz, ``Computational Aspects of Design and Simulation: COSY INFINITY,'' NIM A298, (1990).
%% \bibitem{Eversmann}
%%   D. Eversmann et al., ``New method for a continuous determination of the spin tune in storage rings and implications for precision experiments,'' Phys. Rev. Lett. 115, 094801 (2015)
%% \bibitem{Aksentyev}
%%   A. Aksentyev and Y. Senichev, ``Statistical precision in charged particle EDM search
%%   in storage rings,'' Proceedings IPAC 2017.

%% \bibitem{decoherence}
%%   G. Guidoboni et al. (JEDI Collaboration), ``How to Reach a Thousand-Second in-Plane Polarization Lifetime with 0.97−GeV/c Deuterons in a Storage Ring,'' Phys. Rev. Lett. 117, 054801 (2016)
\end{thebibliography}

%REFERENCES	 
%[1] Canetti L., Drewes M., Shaposhnikov M. (2012). "Matter and Antimatter in the Universe", New Journal of Physics, 14, 095012.
%[2] Aguilar M., Alberti G., Alpat B. et al., (2013) "First Result from the Alpha Magnetic Spectrometer on the International Space Station: Precision Measurement of the Positron Fraction in Primary Cosmic Rays of 0.5 - 350 GeV", Phys. Rev. Lett., 110, 141102.
%[3] Sakharov A. (1967). "Violation of CP Invariance, C Asymmetry, and Baryon Asymmetry of the Universe", Letters to Jounal of Experimental and Theoretical Physics, 5, 24- 26.
%[4] F. J. M. Farley, K. Jungmann, J. P. Miller, W.M. Morse, Y. F. Orlov, B. L. Roberts, Y. K. Semertzidis, A. Silenko  and E. J. Stephenson, New Method of Measuring Electric Dipole Moments in Storage Rings, Phys. Rev. Lett. 93,052001 (2004).
%[5] D. Anastassopoulos et al., “AGS Proposal: Search for a permanent electric dipole moment of the deuteron nucleus at the 10−29 e · cm level”, BNL, 2008.
%[6] M. Berz, “Computational Aspects of Design and Simulation: COSY INFINITY”, NIM A298, (1990).
%[7] Yu.Senichev et al., Investigation of Lattice for Deuteron EDM Ring, Proceedings of ICAP2015, Shanghai, China
%[8] D. Eversmann et al., New method for a continuous determination of the spin tune in storage rings and implications for precision experiments, Phys. Rev. Lett. 115, 094801 (2015)
%[9] A. Aksentyev and Y. Senichev, Statistical precision in charged particle EDM search
%in storage rings, Proceedings IPAC 2017.
%[10] Y. Senichev et al., Spin tune decoherence effects in Electro- and Magnetostatic Structures”, Proceedings of IPAC 2013, Shanghai, China, pp. 2579-2581
%[11]	I. Koop et al., Asymmetric Energy Colliding Ion Beams in the EDM Storage Ring, Proceedings of IPAC 2013, Shanghai, China, pp.1961-1963

\end{document}
