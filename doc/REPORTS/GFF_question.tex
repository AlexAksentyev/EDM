\documentclass{article}
\usepackage[left=2.5cm, right=2cm, top=1.5cm, bottom=2cm]{geometry}

\usepackage{mathtools}
\usepackage{paralist}
\usepackage[utf8]{inputenc}
\usepackage[russian]{babel}
\usepackage[OT1]{fontenc}
\usepackage{paralist}

\newcommand{\SD}[1]{\sigma\left[{#1}\right]}
\newcommand{\Xpct}[1]{\mathrm{E}\left[{#1}\right]}
\newcommand{\avg}[1]{\langle{#1}\rangle}
\newcommand{\W}{\Omega}

\begin{document}

Утверждения по поводу величин, используемых при калибровке ведущего поля:
\begin{enumerate}
\item Ошибка $\SD{\hat\W_y}$ оценки частоты прецессии спина в горизонтальной плоскости $\W_y$
  примерно равна ошибке $\SD{\hat\W_x}$ оценки частоты прецессии спина в вертикальной плоскости $\W_x$:
  \[\SD{\hat\W_y} \approx \SD{\hat\W_x}.\]
  Это обусловлено тем, что:
  \begin{itemize}
  \item данные, на основе которых вычисляются статистики $\hat\W_x$, $\hat\W_y$ обладают одинаковыми свойствами
    (количество точек, временная ширина сэмпла, та же временная структура систематических эффектов и т.д.);
  \item используется одинаковый метод вычисления (т.е. в обоих случаях фитируем одной и той же моделью).
  \end{itemize}
\item Величина ошибки $\SD{\hat\W_x} \sim 10^{-5}$ rad/sec за один филл.
\item Влияние ЭДМ можно заметить только на уровне ошибки $\SD{\hat\W_x}\sim 10^{-7}$ rad/sec,
  для чего надо собирать статистику год.
\item Смена полярности ведущего поля производится на основе статистики одного филла.
\end{enumerate}

В связи с этим возникает вопрос: почему нельзя исключить частоту прецессии в горизонтальной плоскости
из процедуры калибровки?

Вместо
\begin{inparaenum}[(1)]
\item подавления $\W_x$,
\item отведения энергии пучка от резонанса,
\item изменения величины ведущего поля,\footnote{Что само по себе вводит большую ошибку,
  которая никак не калибруется} чтобы сохранить радиус орбиты,
\end{inparaenum}
сразу восстанавливать величину перевёрнутого поля по условию
\[
\hat\W_x^{CCW} = -\hat\W_x^{CW}.
\]

У нас всё равно точность восстановления не превосходит своего частотного эквивалента $10^{-5}$,
т.е. статистическая ошибка многократно превосходит систематический сдвиг засчёт ЭДМ.
Это значит мы не в состоянии систематически компенсировать ЭДМ-эффект при смене полярности
(систематическая ошибка) в принципе. Переход к горизонтальной плоскости не добавляет нам точности,
на сколько я могу судить, он просто смещает калибровочную величину с ненулевого значения на нулевое.

\section{Рассмотрение через ожидание МДМ частоты}
Если рассматривать подстройку магнитного поля с точки зрения МДМ прецессии, то
совокупная частота $\W_{net}$ раскладывается на
\begin{align*}
  \W_{net}^{CW} &= +\W_{MDM}^{CW} + \W_{EDM}^{CW}, \\
  \W_{net}^{CCW} &= -\W_{MDM}^{CCW} + \W_{EDM}^{CCW}.
\end{align*}

Строго говоря, если значение магнитного поля $\avg{B}^{CW}\neq \avg{B}^{CCW}$, то $\W_{EDM}^{CW}\neq\W_{EDM}^{CCW}$,
но пока пренебрежём этим, и будем считать $\W_{EDM}^{CW} = \W_{EDM}^{CCW} \equiv \W_{EDM} = \Xpct{\W_{EDM}}$.

Таким образом, если мы собираем статистику частоты МДМ прецессии в экспериментах,
\begin{align*}
  \tilde\W_{MDM}^{CW} &= \W_{net}^{CW} - \W_{EDM} + \Delta\W, \\
  -\tilde\W_{MDM}^{CCW} &= \W_{net}^{CCW} - \W_{EDM} + \Delta\W, \\
  \Delta\W &\sim N(0, 10^{-5}),
\end{align*}
где $\Delta\W$ это точность,  с которой мы можем выставить частоту в единичном эксперименте.

Тогда, если мы будем стремиться в каждом эксперименте просто приравнивать ожидания
$\Xpct{\W_{net}^{CW}} = -\Xpct{\W_{net}^{CCW}} = \W_0$, то ожидания
\[
\Xpct{\tilde\W_{MDM}^{CW}} = \W_0 -\W_{EDM} \neq \W_0 + \W_{EDM} = \Xpct{\tilde\W_{MDM}^{CCW}},
\]
то есть, в CW экспериментах модуль частоты МДМ прецессии будет отличаться
от модуля МДМ частоты в CCW экспериментах на $2\cdot \W_{EDM}$.

\section{Рассмотрение через таргет-статистику $\hat\W_{EDM}$}
\[
2\cdot \hat\W_{EDM} = \hat\W_{net}^{CW} + \hat\W_{net}^{CCW},
\]
поэтому, если мы изначально в каждом эксперименте будем стремиться делать
$\hat\W_{net}^{CW} = -\hat\W_{net}^{CCW}$, то $\hat\W_{EDM}$ будет тождественно равно 0,
и теряется смысл эксперимента.

\end{document}
