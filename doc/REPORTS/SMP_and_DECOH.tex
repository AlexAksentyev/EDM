\documentclass{article}

\usepackage[left=2cm, right=1.5cm, top=1.5cm, bottom=1.5cm]{geometry}

\usepackage{mathtools}
\usepackage{paralist}
\usepackage[utf8]{inputenc}
\usepackage[russian]{babel}
\usepackage[OT1]{fontenc}

\newcommand{\W}{\Omega}
\newcommand{\w}{\omega}
\newcommand{\const}{\mathrm{const}}
\newcommand{\s}{{ \mathrm{syst} }}

\begin{document}
Изначально статьи
\begin{itemize}
\item Spin Decoherence in the Frequency Domain Method for the EDM Search
\item Spin Motion Perturbation Effect on the EDM Statistic in the Frequency Domain Method
\end{itemize}
были разделены поскольку предполагалось что дисперсия спин-тюнов и прецессия
осей инвариантного спина частиц пучка имеют разные, разделимые эффекты на
поляризацию пучка и нашу способность оценивать частоту её прецессии.

В изначальной концепции не рассматривается поляризация банча, а только спин-вектор одной частицы,
участвующей в бетатронном и синхротронном движениях. Свойства спин-вектора такие:
\begin{itemize}
\item Вертикальная компонента спин-вектора частицы $S_Y = \frac{\W_X}{\W}\cdot \sin(\W\cdot t + \phi_0)$;
\item Из-за декогеренции обе $\W$, $\W_x$ меняются по гармоническому закону.
\end{itemize}

\paragraph{Случай фиксированной оси стабильного спина}
Если не рассматривать движение оси стабильного спина частицы $\bar n$, а принять его фиксированным
замкнутой орбитой, то отношение $\frac{\W_X}{\W} = a_0 = \const$, и нужно фитировать гармонический сигнал
с переменной частотой:
\[
S_Y = a_0\cdot \sin\left(\left[\W_0 + a_\s\sin(\w_\s t + \phi_\s)\right]\cdot t + \phi_0\right).
\]

Отметим следующие факты:
\begin{enumerate}
\item фитирование такого сигнала простой гармонической функцией ведёт к систематической ошибке
  спецификации модели, и соответственно сдвигу оцениваемых параметров;
\item если рассматривать поляризацию, а не спин-вектор одной частицы, систематическая ошибка
  сохранится, как категория, но изменит свой характер, поскольку в пучке будет заполнен приблизительно
  весь спектр возможных частот, генерированный $g(t) = \W_0 + a_\s\sin(\w_\s + \phi_\s)$, т.е. все
  значения частоты будут присутствовать в каждый момент времени;
\item если сделать так, что спин-тюн постоянен на объёме фазового пространства, в который умещается 
  весь пучок, систематическая ошибка исчезает, поскольку это условие эквивалентно $a_\s \rightarrow 0$;
  \item соответственно, проблема решается и для вектора поляризации.
\end{enumerate}

\paragraph{Случай не фиксированной оси стабильного спина}
Если же теперь ввести в рассмотрение изменение наклона оси стабильного спина, при этом
\textbf{предполагая}, что эффект секступольных полей на неё отличается от эффекта на спин-тюн,
т.е. отсутствует выравнивабщий эффект, уравнения меняются:
\begin{align*}
  S_Y &= a_0(t)\cdot \sin(\W_0\cdot t + \phi_0), \\
  a_0(t) &= \W_X(t) / \W_0, \\
  \W_0 &= \const.
\end{align*}

В этом случае снова возникает ошибка мис-спецфикации модели, и соответствующий баяс в оценке её параметров.

Вот эта ошибка и называется ``SMP effect on the EDM statistic in the FD method.'' Принципиально то,
что, если эта ошибка существует, что она нарушает иммунитет Frequency Domain к эффекту геометрической фазы.
Чисто концептуально, конечно, мы меряем частоту, и она постоянна, а значит существует иммунитет;
но практически, мы меряем частоту на основании фазовых данных, а они такого иммунитета не имеют.

Пока что, исследования кажется свидетельствуют о том, что спин-тюн и инвариантная спиновая ось жёстко связаны
между собой, так что выравнивая одну, мы выравниваем и другую. Перекладывая это на язык вектора частоты,
при добавлении к $\vec\W$ некоторой поправки $\Delta\vec\w$, одновременно меняется и направление, и магнитуда
совокупного вектора. Это кажется логичным.

\paragraph{Контент SMP-статьи}
Не кажется возможным формирование условий симуляции таким образом, чтобы исключить дисперсию спин-тюна,
оставив при этом эффект прецессии оси стабильного спина частицы. В свзи с этим, необходимо либо
\begin{inparaenum}[\itshape a\upshape)]
\item изменить содержание статьи о влиянии нарушения спинового движения на конечную статистику FD метода
  на рассмотрение связи между спин-тюном и спиновой осью, а также их свойства в структуре с
  замороженным спином (это линейные графики средних уровней против друг друга, и рисунки 3d-поверхностей
  спин-тюна и оси прецессии на фазовом пространстве), либо
\item провести более сложный анализ, не требуя постоянства частоты прецессии спина.
\end{inparaenum}

В принципе, последний вариант возможен, если абстрагироваться от влияния движения оси на
именно гармоническую модель, и говорить только об изменениях, которые это движение вызывает в самих данных. 

\end{document}
