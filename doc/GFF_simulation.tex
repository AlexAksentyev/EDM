\documentclass[a4paper,14pt]{article}
\usepackage[left=2cm, bottom=1.5cm, right=1.5cm, top=2cm]{geometry}

\usepackage[T1]{fontenc}
\usepackage[utf8]{inputenc}
\usepackage[russian]{babel}

\usepackage{paralist}
\usepackage{amsmath}
\usepackage{xparse}

\newcommand{\Traj}{\mathcal T}
\DeclareDocumentCommand{\Stab}{s}{\mathcal{S}\IfBooleanT{#1}{\vert_{\W_y=0}}}
\newcommand{\Fail}{\mathcal F}
\newcommand{\avg}[1]{\langle {#1} \rangle}
\newcommand{\W}{\Omega}
\DeclareDocumentCommand{\g}{s}{\gamma\IfBooleanT{#1}{_{eff}}}
\newcommand{\nbar}{\bar n}
\newcommand{\CO}{\mathrm{CO}}

\begin{document}

Пусть $\Traj$ обозначает множество всех возможных траекторий частицы в ускорителе. $\Traj = \Stab \bigcup \Fail$,
где $\Stab$ это все стабильные траектории, $\Fail$ это такие траектории, при попадании на одну из которых
частица улетает из пучка.

Первое, что мы делаем, после смены полярности поля --- это подстраиваем его величину так, чтобы частицы
инжектированного пучка попадали на траектории $t\in \Stab$. Это то же самое, что сказать, что выполняется
условие $E + vB = \frac{mv^2}{R}$, и то же самое, что и $\avg{F_L} = 0$.
($\forall t [t\in\Stab \leftrightarrow \avg{F_L} = 0]$.)

Наше второе условие, $\W_y = 0$, выбирает из $\Stab$ подмножество $\Stab*$ траекторий,
для которых спин заморожен в горизонтальной плоскости.

\paragraph{Предположение.}
Предположим, что $\W_y = \W_y(\g*)$ --- инъективная функция,
и значит $\W_y(\g*^1) = \W_f(\g*^2) \rightarrow \g*^1 = \g*^2$. Пространство траекторий делится на
классы эквивалентности по величине эффективного Лоренц-фактора: траектории с одинаковым $\g*$ эквивалентны
с точки зрения спин-динамики, и принадлежат одному классу. Поскольку $\W_y$ инъективная, значит существует
одна гамма эффективное, один класс эквивалентности, при котором $\W_y=0$: $[\W_y=0]\equiv [\g*^0] = \Stab*$.

Если бы мы не использовали секступоли, $\Stab*$ было бы синглетоном (множество с единственным элементом).
Поскольку мы их используем, $\Stab*$ содержит несколько траекторий.

Ранее, мы уже показали, что при использовании секступолей, $\forall t_1,t_2\in\Stab*$:
$\nu_s(t_1) = \nu_s(t_2)$, $\nbar(t_1) = \nbar(t_2)$.

Тогда, чтобы подвтердить валидность калибровочной процедуры, нам нужно показать следующее:
\begin{enumerate}
\item $\Stab*^{CCW} = \Stab*^{CW}$ --- то есть, что и я прямом, и в обратном случае циркуляции пучка,
  $\W_y = 0$ для одних и тех же траекторий (эквивалентно, $\W_y=0$ при одном и том же $\g*$ и в CW, и в CCW
  случаях);
\item $\forall t_1,t_2\in\Stab*^{CCW}$: $\nu_s(t_1) = \nu_s(t_2)$, $\nbar(t_1) = \nbar(t_2)$ ---
  то есть, те же самые секступольные поля подавляют декогеренцию обратного пучка.
\end{enumerate}

\paragraph{Симуляция.}
Для этого нужно всего лишь:
\begin{enumerate}
\item построить график $\nu_s(z),~z\in\{x,y,d\}$ для CW пучка;
\item построить такой же график для CCW пучка;
\item построить их невязку $\epsilon(z)$.
\end{enumerate}

Если невязка мала в широком диапазоне $z$, значит
\begin{inparaenum}[1)]
\item секступольная когеренция работает без изменений для обоих пучков, и
\item спин-тюн/$\g*$ одинаков для обоих пучков, и значит их Спин-Колёса вращаются с одинаковой скоростью.
\end{inparaenum}

Проверять наклоны $\nbar^{CW}$ и $\nbar^{CCW}$ не обязательно, потому что мы и так знаем что
в идеале они $\pm\hat x$, а на самом деле угол наклона определяется точностью, с которой мы выставили $\W_y=0$.

\paragraph{Замечание.}
Единственное что не очень схвачено в методе симуляции выше: в COSY INFINITY система координат связана с
замкнутой орбитой, и на сколько я понимаю, если $\CO^{CW}\neq \CO^{CCW}$, то начала отсчёта графиков в
физическом пространстве будут смещены друг относительно друга. То есть, если $\nu_s(x_0)^{CW} = \nu_s(x_0)^{CCW}$
в данных, то если отсчёт ведётся относительно разных $\CO$, на самом деле будет
$\nu_s(x_0)^{CW} = \nu_s(x_0+\Delta x)^{CCW}$.

В принципе, если $\Delta x$ не очень велико, нам это не слишком мешает, потому что
$x_0\equiv_{\g*} x_0+\Delta x_0$ пока $x_0, x_0+\Delta x \in \Stab*$. В этом вся суть
секступольной когеренции: разные орбиты эквивалентны, и это касается замкнутых орбит, пока они не выходят
за рамки некоторого диапазона. Но было бы неплохо понять
на сколько различаются $\CO$ CW и CCW пучков. Вроде это можно сделать; со слов Артёма, приблуда Розенталя рисует
замкнутые орбиты пучков, и можно видеть если они отличаются для двух пучков.

Хотя, на данный момент, для обращения структуры я использую комбинацию процедур
MR + SMR~\cite[стр.~233]{Eremey:Thesis}, и они по-идее должны сохранять замкнутую орбиту при обращении матриц.

\begin{thebibliography}{9}
\bibitem{Eremey:Thesis}
  Eremey Valetov. FIELD MODELING, SYMPLECTIC TRACKING, AND SPIN DECOHERENCE FOR EDM AND MUON G-2 LATTICES [Internet]. [Michigan, USA]: Michigan State University; Available from: \url{http://collaborations.fz-juelich.de/ikp/jedi/public_files/theses/valetovphd.pdf}
\end{thebibliography}

\end{document}

