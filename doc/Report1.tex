\documentclass{report}


\usepackage[widepage]{../Reports/repsty}
\usepackage[utf8]{inputenc}
\usepackage[T1]{fontenc}
\usepackage[russian]{babel}



\begin{document}
\paragraph{Март}
06.03.17. Моделирование декогеренции сигнала в R.
\paragraph{Май}
11.05.17. Статистическая ошибка.
\begin{itemize}
	\item Статистическая ошибка 1000-секундного сэмпла за год достаточна для достижения статистической погрешности $10^{-9}$ рад/сек в оценке частоты прецессии асимметрии.
	\item Абсолютная ошибка не зависит от частоты.
	\item Возможность выигрыша от использования модуляции.
\end{itemize}
12.05.17. Понял как посчитать спин-матрицу через много оборотов, написал код для получения зависимости X-овой компоненты спина от:
\begin{itemize}
	\item начального отклонения частицы от референсной орбиты.
	\item отклонения по $\gamma$.
\end{itemize}
22.05.17. Вписал в TR (функция трекинга) предыдущий код, чтобы сохранить симплектичность.
\paragraph{Июнь}
09.06.17. Проблема в функции TSP: spin-tune is not an orbital invariant. Спросил Еремея почему, нашёл нарушаемое условие, анализирую структуру DA-вектора из условия в R.
12.06.17. Добавляю в TR поворот ансамбля лучей в горизонтальной плоскости 
29.06.17. Моделирование CW/CCW.
\begin{itemize}
	\item Сформулировал проблему в \emph{решабельном} виде.
	\item Написал код.
\end{itemize}
30.06.17. Код через TR работает долго, хочется использовать функцию, которая считает спин-тюн сразу. Функция TSP даёт спин-тюн как функцию только параметров. Спросил у Еремея как можно модифицировать её, чтобы она давала зависимость от фазовых переменных тоже.
\paragraph{Июль}
01.07.17. Еремей указал на функцию TSS, которая делает именно это. 

04.07.17. Написал код, который считает спин-тюн для системы в прямом и обратном направлении для любого заданного количества частиц. Встречаются проблемы с спин-орбитальным резонансом при высокой точности вычислений --- спин-тюн и $\bar{n}$ не существуют, спин-тюн не орбитальный инвариант.
05--06.07.17. Анализирую написаный код, проверяю уравнение, строю графики, которые можно предоставить Еремею для описания проблемы.

07.07.17. Описываю как я вычисляю спин-тюн, задаю Еремею вопрос почему спин-тюн мал. Почему $\bar{n}_x$ ведёт себя как мы ожидаем --- то есть, не как просто вычислительная погрешность, --- но имеет величину, как если бы она была вычислительным артефактом. 

08.07.17. Еремей говорит что метод нормальных форм плохо работает вблизи резонансов; что спин-орбитальный резонанс, это скорее всего свойство BNL структуры, и что с этим ничего нельзя поделать, разве что написать другой метод. Спрашиваю как он считал спин-тюн, говорит что через TR, и что он вместе со спин-тюном много чего ещё с трекингом делал, поэтому overhead небольшой. Спрашиваю сложно ли написать другой метод, если \emph{не понимаешь что происходит в TSS}. Говорит что TSS ``писал не Мартин,'' и это ``довольно серьёзная вещь.''

10.07.17. Сейчас собираюсь перепроверять код для получения спин-тюна через TR --- почему получаемый спн-тюн в 10 раз меньше ожидаемого.
	
\end{document}