\documentclass[a4paper,14pt]{article}
\usepackage[left=2cm, bottom=1.5cm, right=1.5cm, top=2cm]{geometry}
\usepackage{amsmath}
\usepackage{amssymb}
\usepackage{url}
\usepackage{hyperref}

\newcommand{\avg}[1]{\langle {#1} \rangle}
\newcommand{\xp}[1]{\mathrm{E}\left[{#1}\right]}
\newcommand{\const}{\mathrm{const}}


\begin{document}
\title{Heisenberg principle in relation to a frequency measurement}
\maketitle{}

\section{The Heisenberg relationship}
The Heisenberg principle basically says that, for some observables (called canonically-conjugate),
it is impossible to measure both of them simultaneously with infinite precision.
Examples of such variables are: position and momentum, energy and time, etc.

Let's consider polarization as a function of time $P_y(t) = P_0\sin(\omega\cdot t + \phi)$. A measurement
$P_y(t_i)$ takes a finite amount of time $\Delta t$ to make. If you try to $\Delta t\to 0$, then the error
on $\sigma[P_y(t_i)]\to 1$. The longer you make $\Delta t$, the more precise is the measurement of
polarization. But if it takes $\Delta t$ to make a measurement $P_y(t_i)$, $t_i$ is known to
the precision $\Delta t$. That means time and polarization are related by a Heisenberg-like relationship
\[
\sigma[P]\Delta t \ge \const.
\]

\paragraph{Notes}
\begin{itemize}
\item Actually, in the relationship above $\const = 1\cdot 0 = 0$. (W/o making a measurement,
  you a priori know $P_y$ cannot be less than -1, or greater than 1.)
\item The relationship lacks $\hbar$; but then, the general expression for the Heisenberg relationship
  between operators $A$ and $B$ is $\sigma_A\cdot\sigma_B \ge |\avg{[A,B]}|$, and who says the commutator
  \emph{must} involve $\hbar$? It \emph{could} be zero even.
\end{itemize}
\section{How to account for it}

Our formulas for the standard deviation $\sigma[\hat\omega]$ of the polarization precession frequency estimate
were based on the assumption that the predictor variable (time) is known to infinite precision, and hence
only the dependent variable (polarization) is known with an uncertainty. Realistically it is not,
and the error is actually quite large. Suppose we want to have a 3\% error on the polarization value; it takes
about 2,000 detector counts to register; polarimeter sampling rate is 1 MHz, hence
\[
\Delta t = 2\cdot 10^3/10^6 = 2\cdot 10^{-3} sec.
\]

Take linear regression for example. If we use simple ordinary linear regression on a data set in which
both the dependent and the independent variables are known with an uncertainty,
we'll get biased slope estimates.~\cite[p.~55]{EIP} (Generally, the slope will be underestimated.)

The usual way to deal with this bias is to use an errors-in-variables model.~\cite{EIV-models}

\begin{thebibliography}{9}
\bibitem{EIP}
  \url{http://www.biostat.jhsph.edu/~iruczins/teaching/jf/ch4.pdf}
\bibitem{EIV-models}
  \url{https://en.wikipedia.org/wiki/Errors-in-variables_models}
\end{thebibliography}
\end{document}
  
