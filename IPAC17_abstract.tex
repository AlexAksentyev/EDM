\documentclass{article}
\usepackage{repsty}




\begin{document}
	\title{Model of statistical errors in the search for the deuteron EDM in the storage ring}
	\author{A. Aksentev, IKP, Forschungszentrum J\"ulich, Germany\\ on behalf of the JEDI Collaboration}
	\date{}
	\maketitle

\begin{abstract}
In this work we investigate the standard error of the spin precession frequency estimate in an experiment for the search for the electric dipole moment (EDM) of the deuteron using the polarimeter. The basic principle of polarimetry is the scattering of a polarized beam on a carbon target. Since the number of particles in one fill is limited, we must maximize the utility of the beam. This raises the question of sampling efficiency, as the signal, being an oscillating function, varies in informational content. To address it, we define a numerical measurement model, and compare two sampling strategies (uniform and frequency-modulated) in terms of beam-use efficiency. The upshot is the formulation of the conditions necessary for the effective use of the modulated sampling strategy, and the evaluation of its advantage over the uniform strategy. The simulation results are also used to compare two competing analytical models for the standard error of the frequency estimate.
\end{abstract}
	
\end{document}