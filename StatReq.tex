\documentclass{article}

\usepackage{repsty}


\begin{document}
The questions are:
\begin{enumerate}
	\item How long to measure the signal?
	\item How many measurements per event are optimal?
	\item How congregated about the zero-crossings the measurements should be?
\end{enumerate}

The measurement time is determined by the spin tune decoherence time. This question should be answered first, because it doesn't depend on anything else, and because it is the major factor affecting the standard error of the estimate. We want the events to be as spread out as makes sense statistically. Varying the number of measurements per event won't change the spread, but it will the precision and number of events, as well as their compaction factor.

The two questions that follow both condition the measurement distribution across time. The number of measurements per event determine, with regard to this statistical aspect, the number of events per second, and hence the number of events that can fit within a compaction region. If that number is large, there will be fewer (but more precise) measurements within the high-information regions, and we will have to increase the compaction factor in order to collect enough events, at which point our use of the beam is sub-optimal.
\end{document}